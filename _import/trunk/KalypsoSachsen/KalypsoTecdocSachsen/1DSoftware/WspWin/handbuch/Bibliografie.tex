   \cleardoublepage
   \begin{thebibliography}{99}
      \bibitem{ATV110}           ATV: \emph{Richtlinien f\"{u}r die hydraulische Dimensionierung und den Leistungsnachweis
                                 von  Abwasserkan\"{a}len und -leitungen}
                                 \\ ATV-Arbeitblatt A~110, Aug.~1988
      \bibitem{ATV111}           ATV: \emph{Richtlinien f\"{u}r die hydraulische Dimensionierung und den Leistungsnachweis
                                 von Regenwasserentlastungsanlagen in Abwasserkan\"{a}len und -leitungen}
                                 \\ ATV-Arbeitblatt A~111, Febr.~1994
      \bibitem{ATV118}           ATV: \emph{Richtlinien f\"{u}r die hydraulische Berechnung von Schmutz-, Regen- und
                                 Mischwasserkan\"{a}len}
                                 \\ ATV-Arbeitblatt A~118, 1977
      \bibitem{ATV128}           ATV: \emph{Richtlinien f\"{u}r die Bemessung und Gestaltung von Regenentlastungsanlagen in
                                 Mischwasserkan\"{a}len}
                                 \\ ATV-Arbeitblatt A~128, 1992
      \bibitem{Bleines}          W. Bleines: \emph{Hydraulischer Stau und R\"{u}ckstau bei Durchl\"{a}ssen als Grundlagen ihrer
                                 Bemessung und Konstruktion}
                                 \\ Wasser und Boden Heft~3/1969
      \bibitem{Bollrich}         G. Bollrich: \emph{Technische Hydromechanik Band 1}
                                 \\ Verlag Bauwesen Berlin ,5.~Auflage, 2000
      \bibitem{Breiner1}         H. Breiner: \emph{Neue Gesichtspunkte zur Wasserspiegellagenberechnung bei station\"{a}ren
                                 Abfl\"{u}ssen in offenen Gerinnen}
                                 \\ \"{O}sterreichische Wasserwirtschaft Heft~5/6 1989, S.~122-130 und Heft~3/4 1990, S.~95-105
      \bibitem{Breiner2}         H. Breiner: \emph{Die durch Integration diskretisierten Grundgleichungen zur
                                 eindimensionalen Beschreibung von Abflu{\ss}vorg\"{a}ngen in offenen Gerinnen}
                                 \\ \"{O}sterreichische Wasserwirtschaft Heft~1/2 1990, S.~17-25
      \bibitem{BretschnOezbeck}  H. Bretschneider, T. \"{O}zbeck: \emph{Durchflu{\ss}ermittlung bei geometrisch gegliederten
                                 Gerinnen}
                                 \\ Wasserwirtschaft Heft~4/1997, S.~206-209
      \bibitem{BWK1999}          BWK: \emph{Hydraulische Berechnung von naturnahen Flie{\ss}gew\"{a}ssern Teil~1: Station\"{a}re
                                 Berechnung unter besonderer Ber\"{u}cksichtigung von Bewuchs- und Bauwerkseinfl\"{u}ssen},
                                 \\ Bund der Ingenieure f\"{u}r Wasserwirtschaft, Abfallwirtschaft und Kulturbau, Sept.~1999
      \bibitem{BWK2000}          BWK: \emph{Hydraulische Berechnung von naturnahen Flie{\ss}gew\"{a}ssern -- Grundlagen f\"{u}r
                                 station\"{a}re, eindimensionale Wasserspiegellagenberechnungen}
                                 \\ Bund der Ingenieure f\"{u}r Wasserwirtschaft, Abfallwirtschaft und Kulturbau, Bericht~1/2000,
                                 M\"{a}rz~2000
      \bibitem{Chow}             Ven Te Chow: \emph{Open Channel Hydraulics}
                                 \\ Mc Graw-Hill Book Co., New York 1959
      \bibitem{DtForschungsgem}  Deutsche Forschungsgemeinschaft: \emph{Hydraulische Probleme beim naturnahen Gew\"{a}sserausbau}
                                 \\ VCH Verlagsgesellschaft, Weinheim~1987
      \bibitem{DVWK1991}         DVWK: \emph{Hydraulische Berechnung von Flie{\ss}gew\"{a}ssern},
                                 \\ Merkbl\"{a}tter zur Wasserwirtschaft Heft~220/1991, Verlag Paul Parey 1991
      \bibitem{EulerKoussis}     G. Euler, A. Koussis: \emph{Die Berechnung von Hochwasserabl\"{a}ufen mit N\"{a}herungsverfahren
                                 und ihre Anwendung}
                                 \\ Wasserwirtschaft Heft~8/1973
      \bibitem{EulerWackermann}  G. Euler, R. Wackermann: \emph{Niederschlag-Abflu{\ss}-Modelle am Beispiel der Nidda}
                                 \\ DVWK-Schriftenreihe Heft~51, Bonn 1980
      \bibitem{FelkelCanisius}   K. Felkel, P. Canisius: \emph{Rechenautomatenprogramm zur Spiegelberechnung f\"{u}r ausufernde
                                 Hochw\"{a}sser}
                                 \\ Wasserwirtschaft Heft~8/1967, S.~308-314
      \bibitem{Kaiser}           W. Kaiser: \emph{Flie{\ss}widerstandsverhalten in Gerinnen mit durchstr\"{o}mten Ufergeh\"{o}lzzonen}
                                 \\ Wasserbau-Mitteilungen, TH Darmstadt, Nr.~23, Sept.~1984
      \bibitem{Knapp}            F. H. Knapp: \emph{Ausflu{\ss}, \"{U}berfall und Durchflu{\ss} im Wasserbau}
                                 \\ Verlag G. Braun, Karlsruhe~1960
      \bibitem{Knauf1}           D. Knauf: \emph{Flie{\ss}tiefen und Durchfl\"{u}sse f\"{u}r gegliederte Querschnitte}
                                 \\ Korrespondenz Abwasser, Heft~1/1989, S.~37-41
      \bibitem{Knauf2}           D. Knauf: \emph{Flie{\ss}wechselbestimmung bei Abfl\"{u}ssen in gegliederten Querschnitten}
                                 \\ Wasser und Boden, Heft~12/1996
      \bibitem{KnaufKoennemann}  D. Knauf, N. K\"{o}nnemann: \emph{Abflu{\ss}verteilung und Flie{\ss}zustand in gegliederten
                                 Gerinneprofilen}
                                 \\ Wasserwirtschaft Heft~3/1977, S.~61-64
      \bibitem{KnaufMerzdorf}    D. Knauf, F. H. Merzdorf: \emph{Anwenderhandbuch WSPLWA-83}
                                 \\ Eigenverlag LWA-NRW, D\"{u}sseldorf~1983
      \bibitem{Kradolfer}        W. Kradolfer: \emph{Berechnung des Normalabflusses in Gerinnen mit einfachen und
                                 gegliederten Querschnitten}
                                 \\ Mitteilungen der Versuchsanstalt f\"{u}r Wasserbau, Hydrologie und Glaziologie,
                                 ETH~Z\"{u}rich, Nr.~65, Z\"{u}rich~1983
      \bibitem{Lautrich}         R. Lautrich: \emph{Tabellen und Tafeln zur hydraulischen Berechnung von Druckrohrleitungen}
                                 \\ Verlag Wasser und Boden, Hamburg 1969
      \bibitem{Lindner}          K. Lindner: \emph{Der Str\"{o}mungswiderstand von Pflanzenbest\"{a}nden}
                                 \\ Mitteilungen aus dem Leichtwei{\ss} Institut f\"{u}r Wasserbau der TU Braunschweig,
                                    Heft~25, 1982
      \bibitem{LWA-NRW}          LWA-NRW: \emph{Flie{\ss}gew\"{a}sser-Richtlinie f\"{u}r naturnahen Ausbau und Unterhaltung}
                                 \\ Eigenverlag LWA-NRW, D\"{u}sseldorf~1980
      \bibitem{Mertens}          M. Mertens: \emph{Zur Frage hydraulischer Berechnung naturnaher Flie{\ss}gew\"{a}sser}
                                 \\ Wasserwirtschaft Heft~4/1989, S.~170-179
      \bibitem{Merzdorf}         F. H. Merzdorf: \emph{Erfassung und Begrenzung der \"{U}berschwemmungsgebiete im Lande
                                 Nordrhein-Westfahlen}
                                 \\ Wasser und Boden Heft~9/1976, S.~223-226
      \bibitem{Naudascher}       E. Naudascher: \emph{Hydraulik der Gerinne und Gerinnebauwerke}
                                 \\ Springer Verlag, Wien 1987
      \bibitem{NaudaschMedlarz}  E. Naudascher, H. J. Medlarz: \emph{Hydrodynamic Loading and Backwater Effect of Partially
                                 Submerged Bridges}
                                 \\ Journal Hydraulic Research, Vol.~21, No.~3, 1983
      \bibitem{Nuding}           A. Nuding: \emph{Flie{\ss}widerstandsverhalten in Gerinnen mit Ufergeb\"{u}sch}
                                 \\ Wasserbau-Mitteilungen, TH~Darmstadt, Nr.~35, Nov.~1991
      \bibitem{Pasche}           E. Pasche: \emph{Turbulenzmechanismen in nat\"{u}rlichen Flie{\ss}gew\"{a}ssern und die M\"{o}glichkeit
                                 ihrer mathematischen Erfassung}
                                 \\ Dissertation RWTH Aachen, 1984
      \bibitem{PressSchroeder}   H. Press, R. Schr"oder: \emph{Hydromechanik im Wasserbau},
                                 \\ Verlag W. Ernst u. Sohn, Berlin 1966
      \bibitem{RosemannVedral}   H. J. Rosemann, J. Vedral: \emph{Das Kalinin-Miljukov-Verfahren zur Berechnung des
                                 Ablaufs von Hochwasserwellen}
                                 \\ Schriftenreihe der Bayrischen Landesstelle f\"{u}r Gew\"{a}sserkunde Heft~6/1971, M\"{u}nchen
      \bibitem{Schmidt}          M. Schmidt: \emph{Die Berechnung unvollkommener \"{U}berf\"{a}lle}
                                 \\ Wasserwirtschaft Heft~4/1957, S.~174-178
      \bibitem{SchroederJokiel}  M. Schr\"{o}der, C. Jokiel: \emph{ESNA-Manual for Users and Developers}
                                 \\ RWTH~Aachen, Sept.~1993
      \bibitem{SchroederRCM1}    R. C. M. Schr\"{o}der: \emph{Str\"{o}mungsberechnung im Bauwesen Teil~II: Interstation\"{a}re Str\"{o}mungen}
                                 \\ Bauingenieurpraxis, Verlag W. Ernst und Sohn, Berlin 1972
      \bibitem{SchroederRCM2}    R. C. M. Schr\"{o}der: \emph{Technische Hydraulik, Kompendium f\"{u}r den Wasserbau}
                                 \\ Springer-Verlag Berlin u.a., 1994
      \bibitem{SchroederW}       W. Schr\"{o}der: \emph{Wasserbau},
                                 \\ Abschnitt~13 in Schneider Bautabellen, Werner Verlag, D\"{u}sseldorf~1992,
                                 10. Auflage, S. 13.22-13.26
      \bibitem{SchroederW2}      W. Schr\"{o}der: \emph{Grundlagen des Wasserbaus}
                                 \\ 4. Auflage, Werner Verlag, D\"{u}sseldorf 1999
      \bibitem{SchroederNuding1} W. Schr\"{o}der, A. Nuding: \emph{Spiegellinienberechnung f\"{u}r einen Wildbach mit Geh\"{o}lzufern}
                                 \\ Wasserwirtschaft Heft~9/1986, S.~388-395
      \bibitem{SchroederNuding2} W. Schr\"{o}der, A. Nuding: \emph{Flie{\ss}widerstand von Baum- und Buschufern}
                                 \\ Wasser u. Boden, Heft~11/1992, S.727-730
      \bibitem{Schumacher}       F. Schumacher: \emph{Zur Durchflu{\ss}berechnung gegliederter, naturnah gestalter
                                 Flie{\ss}gew\"{a}sser}
                                 \\ Mitt. Nr.~127, TU~Berlin 1995
      \bibitem{Schwarze}         H. Schwarze: \emph{Erweiterung des Anwendungsbereiches der Rehbock'schen
                                 Br\"{u}ckenstaugleichung auf Trapezquerschnitte}
                                 \\ Mitteilung des Franzius-Institutes Heft~33, Hannover 1969
      \bibitem{SeusUslu}         G. Seus, O. Uslu: \emph{Berechnung der Wasserspiegellagen bei station\"{a}r ungleichf\"{o}rmigem
                                 Abflu{\ss} in nat\"{u}rlichen Gerinnen und die Optimierung der Flie{\ss}beiwerte}
                                 \\ in Elektronische Berechnung von Rohr- und Gerinnestr\"{o}mungen, herausgeg. von W. Zielke,
                                 Erich Schmidt Verlag, 1974
      \bibitem{Timm}             J. Timm: \emph{Hydraulisches Berechnen}
                                 \\ Teubner Verlag, Stuttgart 1970
      \bibitem{Watts}            Watts et al: \emph{Variation of $\alpha$- and $\beta$-Values in a Lined Open Channel}
                                 \\ Journal of Hydraulics Division ASCE, Nov 1976, p. 217-234
      \bibitem{WeingaertnerHehn} I. Weing\"{a}rtner, B. Hehn: \emph{Erfahrungen bei der Neuermittlung von
                                 \"{U}berschwemmungsgebieten in NRW}
                                 \\ Wasser und Boden Heft~10/1979, S.~290-294
   \end{thebibliography}