\cleardoublepage
%%%%%%%%%%%%%%%%%%%%%%%%%%%%%%%%%%%%%%%%%%%%%%%%%%%%%%%%%%%%%%%%%%%%%%%%%%%%%%%%%%%%%%%%%%%%%%%%%%%%%%%%%%%%%%%%%%%%%%%%%%%%
\chapter{Vorwort}
%%%%%%%%%%%%%%%%%%%%%%%%%%%%%%%%%%%%%%%%%%%%%%%%%%%%%%%%%%%%%%%%%%%%%%%%%%%%%%%%%%%%%%%%%%%%%%%%%%%%%%%%%%%%%%%%%%%%%%%%%%%%

Die Gef\"{a}hrdung unserer Siedlungsr\"{a}ume durch Hochwasser, aber auch die hohe Bedeutung, die unsere Flie{\ss}gew\"{a}sser f\"{u}r eine
intakte Umwelt haben, fordern vom Ingenieur vielf\"{a}ltige Ma{\ss}nahmen am Gew\"{a}sser, deren Wirkung sicher beurteilt werden mu{\ss}.
Hierzu geh\"{o}ren sowohl Hochwasserschutzma{\ss}nahmen als auch die Verbesserung der \"{o}kologischen Verh\"{a}ltnisse. Gro{\ss}e Bedeutung
kommt dabei der richtigen Erfassung und Bewertung der str\"{o}mungsphysikalischen Vorg\"{a}nge zu \cite{BWK1999}.

Die station\"{a}re Wasserspiegellagenberechnung auf Basis eines eindimensionalen Str\"{o}\-mungsmodells stellt in diesem
Zusammenhang ein zentrales Instrument der wasserwirtschaftlichen Praxis dar. Im Gegensatz zur Anwendung komplexerer
Modelle ist der eindimensionale Spiegelliniennachweis insbesondere dann gefragt, wenn es um einfacher handhabbare und
leichter nachvollziehbare Berechnungsverfahren geht. Eine zweidimensionale Berechnung wird erst dann empfohlen, wenn eine
rechnerische Vereinfachung der in Wirklichkeit dreidimensionalen Str\"{o}mung auf ein eindimensionales Problem keine
hinreichende Genauigkeit mehr liefert. Dies trifft beispielsweise bei stark heterogenen Gew\"{a}ssern zu, wo konvektive
Beschleunigungstherme an Dominanz gewinnen \cite{DVWK1991}.

Aber auch das scheinbar so einfache Problem der Umsetzung von Bemessungsabfl\"{u}ssen in Wasserst\"{a}nde im eindimensionalen
Modell wirft heute noch viele ungel\"{o}ste Fragen auf, die wegen ihrer Komplexit\"{a}t noch lange Gegenstand der
wissenschaftlichen Forschung bleiben werden. Oft sind dies Fragen, die auch nicht durch mehrdimensionale Modelle oder
finite Elemente gel\"{o}st werden k\"{o}nnen, weil Messungen oder Grundlagenuntersuchungen schlicht fehlen (z.B. Abflu{\ss}aufteilung
bei teilweise durch- und \"{u}berstr\"{o}mten Br\"{u}cken, schie{\ss}ender Abflu{\ss}).

Der einfacheren Handhabbarkeit eindimensionaler Modelle sind aber h\"{a}ufig insofern Grenzen gesetzt, als eine Vielzahl
geometrischer und hydraulischer Daten anfallen, die es zu erfassen und formatgerecht zu archivieren gilt. Aus diesen
\"{U}berlegungen heraus besteht \wspwin{} neben dem eigentlichen Berechnungsprogramm aus einer nach modernen Gesichtspunkten
der Informationsverarbeitung konzipierten Windows-Oberfl\"{a}che, die eine vereinfachte Dateneingabe in dialogorientierten,
grafisch unterst\"{u}tzten Masken erm\"{o}glicht.

Mit dem Programmsystem \wspwin{} wird ein Berechnungsinstrument zur Verf\"{u}gung gestellt, das eine leichtere und besser
fundierte Bearbeitung hydraulischer Fragestellungen erm\"{o}glichen soll. Mit dem seit den 60er Jahren stets
weiterentwickelten Berechnungsprogramm der Firma PSW-Knauf und der in den letzten 5~Jahren st\"{a}ndig verbesserten und
erweiterten Benutzeroberfl\"{a}che der Firma BCE steht dem Anwender nun ein abgerundetes, integriertes Programmsystem zu
Verf\"{u}gung, das es erm\"{o}glicht, eine Vielzahl hydraulischer Fragestellungen komfortabel anhand fundierter Berechnungsans\"{a}tze
zu bearbeiten.

Das Programmsystem erm\"{o}glicht den Einsatz verschiedener Berechnungsverfahren, um durch Vergleichsrechnung eine
Erweiterung des Erfahrungshorizontes zu erm\"{o}glichen. Durch die Vielzahl der M\"{o}glichkeiten kann das Programm als sehr
komplex bezeichnet werden. Eine Benutzerf\"{u}hrung in der Windows-Oberfl\"{a}che sowie das vorliegende Handbuch sollen helfen,
die Anwendung einfacher zu gestalten.

Das Handbuch wurde dementsprechend so konzipiert, da{\ss} zun\"{a}chst in einer Einf\"{u}h\-rung wichtige immer wiederkehrende
Begriffe erl\"{a}utert und die Voraussetzungen f\"{u}r die Durch\-f\"{u}hrung einer Spiegellinienberechnung gekl\"{a}rt werden. Nach den
Installationshinweisen wird eine Einf\"{u}hrung in die Handhabung des Programms anhand der Bearbeitung eines kompletten
Beispielprojektes mit Normalprofilen gegeben und dabei die wichtigsten f\"{u}r die Projektbearbeitung relevanten Optionen
erl\"{a}utert. Illustriert durch Beispiele und inhaltlich sortiert behandelt das n\"{a}chste Kapitel Sonderprofile wie Br\"{u}cken,
Wehre, Durchl\"{a}sse etc., so da{\ss} hier gezielt nachgeschlagen werden kann. Die einzelnen Abschnitte gliedern sich jeweils in
einen theoretischen Teil, einen Teil, der die Dateneingabe behandelt und schlie{\ss}lich ein Beispiel. Nachdem die wichtigsten
Optionen bekannt sind, gibt das darauffolgende Kapitel Auskunft \"{u}ber weiterreichende Programmfunktionen. Auch hier kann
gezielt nachgeschlagen werden, wenn eine spezielle Frage auftritt. Reine Zusatzmodule, die f\"{u}r die
Wasserspiegellagenberechnung nicht zwingend n\"{o}tig sind, sind schlie{\ss}lich im letzten Kapitel erl\"{a}utert. Es folgen
Fehlerhinweise und Literaturangaben.

\subsubsection{Einsatzm\"{o}glichkeiten}
\begin{itemize}
   \item Berechnung von \"{U}berschwemmungsgrenzen
   \item Nachweis des \"{U}berflutungsrisikos
   \item Nachweis der Auswirkungen von Bauma{\ss}nahmen
   \item Nachweis der Hochwassersicherheit
   \item Quantifizierung des Str\"{o}mungsfeldes
   \item Beurteilung von Ein- und Ausleitungen
   \item Nachweis der Str\"{o}mung in und an Bauwerken im Gew\"{a}sser
   \item Leistungsnachweis bzw. Bemessung und Gestaltung von Gew\"{a}sserprofilen, Kreu\-zungsbauwerken, Sonderbauwerken und Flutmulden
   \item Ermittlung des Retentionsraums
   \item Ermittlung der Einstauh\"{a}ufigkeit
\end{itemize}
\subsubsection{Berechnungsprogramm}
\begin{itemize}
   \item Berechnung von Abflu{\ss}kurven, Ausuferungsabfl\"{u}ssen, Grenztiefen und Normalabflu{\ss}tiefen
   \item alle Profilformen: Kompaktquerschnitte, Vorlandprofile, geschlossene und auskragende Profile
   \item Br\"{u}cken, Durchl\"{a}sse, Wehre, Verzweigungen
   \item Flie{\ss}formeln \autor{Manning-Strickler} und \autor{Colebrook-White}, Formbeiwerte nach \autor{Marchi}
   \item Rauheitseichung
   \item Ans\"{a}tze f\"{u}r \"{o}rtliche Einzelverluste
   \item Ber\"{u}cksichtigung durchstr\"{o}mten Bewuchses
   \item station\"{a}re Wasserspiegelberechnung f\"{u}r naturnahe Flie{\ss}gew\"{a}sser unter Ber\"{u}cksichtigung von Bewuchsstrukturen und Bauwerkseinfl\"{u}ssen
   \item str\"{o}mender und schie{\ss}ender Abflu{\ss}
   \item Berechnungsansatz f\"{u}r M\"{a}anderstr\"{o}mungen
\end{itemize}
\subsubsection{Oberfl\"{a}che}
\begin{itemize}
   \item dialogorientierte Eingabemasken, Benutzerf\"{u}hrung durch s\"{a}mtliche Arbeitsschritte
   \item tabellarische und grafische Anzeige der Berechnungsergebnisse mit Druckfunktion
   \item Dateimanager (Dateiselektion \"{u}ber Schl\"{u}sselbegriffe)
   \item redundanzminimierte Datenspeicherung
   \item Erstellung von Plotfiles im dxf-Format (Auto-CAD)
   \item Abflu{\ss}- und Einzelverlustdatei, Eingabe von Wasserspiegelfixierungen
   \item alphanumerischer und grafisch-interaktiver Editor mit Zoomfunktion f\"{u}r L\"{a}ngsschnitte
         und Querprofile
   \item Zusatzfunktionen: Rauheitsdatenbank, Massenberechnung, Profilinterpolation
   \item Schnittstellen zu Fremdprogrammen
   \item zeitsparende Berechnung im Stapelbetrieb
   \item automatischer Eintrag des berechneten Wasserspiegels in die Querprofile
   \item Projekte, Zust\"{a}nde und Profile kopieren
   \item Aufruf von Auswerteprogrammen
\end{itemize}

\wspwin{} wird inzwischen in sechs Bundesl�ndern (Bayern, Niedersachsen, Nordrhein-Westfahlen, Sachsen, Schleswig-Holstein, Th�ringen) als Standardwerkzeug von Seiten der Landes�mter f�r Um\-welt\-schutz/Wasserwirtschaft sowie deren nachgeordneten Dienststellen eingesetzt. Als Weiterentwicklung ist 1999 das Plotprogramm WSPWIN-Plotter hinzugekommen. Es er\-m�g\-licht die komfortable grafisch-interaktive Gestaltung von L�ngs- und Querprofil\-zeich\-nun\-gen ohne den Einsatz eines zus�tzlichen CAD-Programms.
Das in 2001 hinzugekommene Programm \wspwin{}-Mapper erm�glicht die Ein- und Ausgabe hy\-draulischer Daten im Lageplan. Durch Verschneidung der berechneten Wasserspiegel mit einem digitalen Gel�ndemodell entstehen pr�zise �berschwemmungspl�ne, f�r die eine Viel\-zahl von Gestaltungsm�glichkeiten bereitgestellt werden. Der Einsatz eines zus�tzlichen geo\-gra\-fischen Informationssystems ist nicht n�tig.
Wir gehen da\-von aus, dass die Entwicklung auch in den folgenden Jahren fortgesetzt wird.
\\
\\
Dipl.-Math. G. Belger\\
Bj�rnsen Beratende Ingenieure GmbH
\\
\\
Prof. Dr.-Ing. D. Knauf\\
PSW Knauf, Seeheim-Jugenheim
\\
\\
aufgestellt: Oktober 1997, 2. Auflage: April 1999, 3. Auflage M�rz 2002
