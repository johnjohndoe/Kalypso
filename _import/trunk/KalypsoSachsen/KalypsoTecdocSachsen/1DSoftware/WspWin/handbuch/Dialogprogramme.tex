\graphicspath{{c:/latex/diplom/handbuch/Dialogprogramme/eps/}} \cleardoublepage
%%%%%%%%%%%%%%%%%%%%%%%%%%%%%%%%%%%%%%%%%%%%%%%%%%%%%%%%%%%%%%%%%%%%%%%%%%%%%%%%%%%%%%%%%%%%%%%%%%%%%%%%%%%%%%%%%%%%%%%%%%%%

\chapter{Dialogprogramme}
\label{Dialogprogramme}

%%%%%%%%%%%%%%%%%%%%%%%%%%%%%%%%%%%%%%%%%%%%%%%%%%%%%%%%%%%%%%%%%%%%%%%%%%%%%%%%%%%%%%%%%%%%%%%%%%%%%%%%%%%%%%%%%%%%%%%%%%%%

%%%%%%%%%%%%%%%%%%%%%%%%%%%%%%%%%%%%%%%%%%%%%%%%%%%%%%%%%%%%%%%%%%%%%%%%%%%%%%%%%%%%%%%%%%%%%%%%%%%%%%%%%%%%%%%%%%%%%%%%%%%%
\section{Inhalt}
%%%%%%%%%%%%%%%%%%%%%%%%%%%%%%%%%%%%%%%%%%%%%%%%%%%%%%%%%%%%%%%%%%%%%%%%%%%%%%%%%%%%%%%%%%%%%%%%%%%%%%%%%%%%%%%%%%%%%%%%%%%%

Das Programm HYDRA ist ein Dialog-Programm zur Berechnung der hydraulischen Leistungsf\"{a}higkeit von offenen und
geschlossenen Flie{\ss}querschnitten. Der Anwendungsbereich entspricht etwa dem Ziel der bekannten Hilfstabellen zur L\"{o}sung
wasserwirtschaftlicher Aufgaben, wobei hier die Unterst\"{u}tzung durch PC-Rechner und der Ersatz nomografischer L\"{o}sungen
durch genauere Rechenverfahren im Vordergrund stehen. Grunds\"{a}tzlich werden die folgenden M\"{o}glichkeiten der Fragestellung
behandelt:
\begin{itemize}
   \item Die Geometrie steht durch einen bekannten Wasserstand fest und der zugeh\"{o}rige Abflu{\ss} im Gerinne wird gesucht.
   \item Der Abflu{\ss} f\"{u}r das Gerinne wird vorgegeben und der Wasserstand bzw. die Flie{\ss}tiefe ist gesucht.
\end{itemize}
S\"{a}mtliche Eingaben zur Definition der Regelprofile (Rechteck, allgemeines Trapez mit unterschiedlichen Rauheiten,
Kreisquerschnitte, Maul- oder Eiprofile) erfolgen im gef\"{u}hrten Dialog.

Die hydraulische Berechnung kann nach \autor{Manning-Strickler} oder \autor{Prandtl-Colebrook} durchgef\"{u}hrt werden.
Hierbei erfolgen alle Berechnungen (gleichg\"{u}ltig ob geometrischer oder hydraulischer Art) analytisch ohne Interpolationen
und Hilfsdiagramme. Dabei sind auch Teilf\"{u}llungszust\"{a}nde von geschlossenen Profilen zugelassen, allerdings entsprechend
der aktuellen ATV-Richtlinie~\cite{ATV111} ohne N\"{a}herung nach \autor{Thormann} (sog. \afz{Luftreibungseinflu{\ss}}).

Das Programmsystem ist modular aufgebaut, d.h. f\"{u}r jede Problemstellung gibt es ein eigenes Unterprogramm, das aus einem
Auswahlmen\"{u} aufgerufen wird. Die Auswahlmen\"{u}s sind hierarchisch nach einer Baumstruktur aufgebaut. Die Men\"{u}steuerung
erfolgt mit den Steuertasten, durch Eingabe der Programmnummer oder durch Eingabe des ersten Programmbuchstabens. Mit der
Taste \schalter{Esc} kann jeweils in das Vormen\"{u} gewechselt oder das Programm beendet werden. Zu jedem Unterprogramm sind
Informationen abrufbereit gespeichert. Der Aufruf erfolgt durch den Buchstaben \schalter{I} mit anschlie{\ss}ender
Programm\-auswahl oder durch die Taste \schalter{F1}.

\clearpage
%%%%%%%%%%%%%%%%%%%%%%%%%%%%%%%%%%%%%%%%%%%%%%%%%%%%%%%%%%%%%%%%%%%%%%%%%%%%%%%%%%%%%%%%%%%%%%%%%%%%%%%%%%%%%%%%%%%%%%%%%%%%
\section{Hinweise zur Anwendung der Dialogprogramme}
%%%%%%%%%%%%%%%%%%%%%%%%%%%%%%%%%%%%%%%%%%%%%%%%%%%%%%%%%%%%%%%%%%%%%%%%%%%%%%%%%%%%%%%%%%%%%%%%%%%%%%%%%%%%%%%%%%%%%%%%%%%%

Alle Dialogprogramme sind streng nach Dateneingaben und Ergebniswerten gegliedert. Die Daten und Ergebnisse k\"{o}nnen auf
Wunsch gespeichert und f\"{u}r eine sp\"{a}tere Variation in einem Dateiauswahl-Men\"{u} wieder abgerufen werden.

Bei Programmstart wird zun\"{a}chst immer gefragt, ob vorhandene eigene Daten als Vorbesetzung gew\"{a}hlt werden sollen. Bei
Eingabe \schalter{N} werden programmintern festgelegte Vorgabewerte angezeigt, die beliebig ver\"{a}ndert werden k\"{o}nnen. Zur
Definition von Eingabegr\"{o}{\ss}en k\"{o}nnen \"{u}ber das Hilfemen\"{u} entsprechende Skizzen (z.B. der Profilform) abgerufen werden. Nach
Anzeige der Ergebnisse erfolgen immer folgende Fragen:
\begin{itemize}
   \item sollen Eingabedaten in eine Datei geschrieben werden?
   \item sollen Ergebnisse in eine Datei geschrieben werden?
   \item sollen Ergebnisse gedruckt werden?
\end{itemize}
Werden die ersten beiden Fragen mit \schalter{J} beantwortet, wird der Dateiname abgefragt: zun\"{a}chst ohne Pfadangabe und
Extension, anschlie{\ss}end werden Standard-Pfadangabe
\begin{quote}
   z.B.: \datei{...\textbackslash HYDRA\textbackslash ROHR\textbackslash}
\end{quote}
und Standard-Extension (z.B. \datei{*.QKI}) vom Programm hinzugef\"{u}gt und erneut angezeigt. Die letzte Dateianzeige kann
beliebig ver\"{a}ndert werden. Zu beachten ist lediglich, da{\ss} eine Dateianzeige im Auswahlfenster nur m\"{o}glich ist, wenn die
Standardextensionen verwendet werden. Die Ergebnisdateien k\"{o}nnen vor der Druckausgabe editiert werden (z.B. durch
Er\-g\"{a}n\-zung zus\"{a}tzlicher \"{U}berschriften oder sonstige Hinweise). Der Ausdruck erfolgt anschlie{\ss}end \"{u}ber das
DRUCK-Hilfsprogramm mit Umsetzung der ggf. vorhandenen Druck-Punktbefehle.


\clearpage
\subsection{Das Hauptmen\"{u}}
Beim Aufruf der Datei \datei{HYDRA.EXE} aus dem Verzeichnis \datei{...\textbackslash HYDRA} erscheint das folgende Men\"{u}
zur Ansteuerung der einzelnen Unterprogramme: \\
\begin{minipage}{1.0\textwidth}
   \footnotesize
   \begin{alltt}


                    \doublebox{\parbox{5.5cm}{        H Y D R A \\ (c.)S O B A U - P S W 1999}}

                              Hauptmen\"{u}
______________________________________________________________________
                           1  -  Gerinne
                           2  -  Rohre
                           3  -  \"{U}berf\"{a}lle
                           4  -  Wasserspiegel
                           5  -  ATV-Richtlinien
                           6  -  Kanalnetze
                           7  -  Stra{\ss}enentw\"{a}sserung
                           8  -  Hilfsprogramme

                           9  -  Information


      Help : F1     Drucker : wie eingestellt      Ende : ESC oder F3
______________________________________________________________________

   \end{alltt}
   \normalsize
\end{minipage}



\subsection{Unterprogramm Gerinne}

Durch die Auswahl von \schalter{1} bzw. \schalter{G} im Hauptmen\"{u} wird das Programm \datei{GERINNE.EXE} aufgerufen. Hier
stehen Ihnen in einem Men\"{u} folgende Optionen zur Verf\"{u}gung: \\
\begin{minipage}{1.0\textwidth}
   \footnotesize
   \begin{alltt}


                             H Y D R A
                     (c.)S O B A U - P S W 1999
                  ABFLUSS GERINNE nach  STRICKLER
______________________________________________________________________
                           1  -   Q   Rechteck/Trapez
                           2  -   KST-Wert nach Einstein
                           3  -   Rinnenprofil
                           4  -   Trapez mit Mulde
                           5  -   Vorlandprofil
                           6  -   Sonderprofil
                           7  -   Weitere Funktionen

                           9  -   Information

      Help : F1                                    Ende : Esc oder F3
______________________________________________________________________

   \end{alltt}
   \normalsize
\end{minipage}
\clearpage
\"{U}ber den Men\"{u}punkt \afz{Weitere Funktionen} kann ein weiteres Men\"{u} abgerufen werden: \\
\begin{minipage}{1.0\textwidth}
   \footnotesize
   \begin{alltt}


                    \doublebox{\parbox{5.5cm}{        H Y D R A \\ (c.)S O B A U - P S W 1999}}

                     SONDERFUNKTIONEN f\"{u}r GERINNE
_____________________________________________________________________
                           1  -  Hnorm Rechteck/Trapez
                           2  -  Grenztiefe  Trapez
                           3  -  Stauweiten
                           4  -  Bepflanzte Flutmulde
                           5  -  Polygonzug - Profile
                           6  -  Q_Durchla{\ss}
                           7  -  Aufstau Durchla{\ss}

                           9  -  Information


      Help : F1                                    Ende : Esc oder F3
______________________________________________________________________

   \end{alltt}
   \normalsize
\end{minipage}

Das Dialogprogramm wurde f\"{u}r die Berechnung nat\"{u}rlicher Vorfluter entwickelt. Voraussetzung f\"{u}r seine Anwendung ist ein
ungest\"{o}rter Abflu{\ss} (Energieliniengef\"{a}lle = Wasserspiegelgef\"{a}lle = Sohlgef\"{a}lle). Dabei stehen folgende Profilformen zur
Verf\"{u}gung:
\begin{itemize}
   \item Rechteck oder unsymmetrisches Trapez
   \item Unsymmetrisches Trapez mit Rinne
   \item Trapez mit Vorl\"{a}ndern
\end{itemize}
Die Berechnung erfolgt auf der Grundlage der Flie{\ss}formel von \autor{Manning-Strickler} unter Ber\"{u}cksichtigung des
Geschwindigkeitsbeiwertes bei gegliederten Querschnitten (N\"{a}herung nach \autor{Posey}~\cite{ATV110}). F\"{u}r prismatische
Gerinne k\"{o}nnen weiterhin die Stauweiten von Stauanlagen und die Senkungsweiten bei Abst\"{u}rzen aus der Differentialgleichung
der Spiegellinie nach \autor{R\"{u}hlmann}~\cite{PressSchroeder} abgesch\"{a}tzt werden. Die Bestimmung der Flie{\ss}fl\"{a}chen erfolgt
mit Hilfe des \autor{Gau{\ss}}'schen Algorithmus f\"{u}r die Fl\"{a}chenberechnung. Mit den Unterprogrammen \afz{Q\_Durchla{\ss}} bzw.
\afz{Aufstau Durchla{\ss}} lassen sich nicht kreisf\"{o}r\-mi\-ge D\"{u}ker berechnen.


\clearpage
\subsection{Unterprogramm Rohre}

Die hydraulische Leistungsf\"{a}higkeit von Rohr- und D\"{u}kerleitungen wird nach dem Flie{\ss}gesetz von \autor{Prandtl-Colebrook}
berechnet. Der Programmaufruf erfolgt im Hauptmen\"{u} \"{u}ber \schalter{2}. \\
\begin{minipage}{1.0\textwidth}
   \footnotesize
   \begin{alltt}


                    \doublebox{\parbox{5.5cm}{        H Y D R A \\ (c.)S O B A U - P S W 1999}}

                       ABFLUSSLEISTUNG  ROHRE
______________________________________________________________________
               Kreisrohr   1  -  Abflu{\ss} Q
                           2  -  Jerf
                           3  -  Grenz h
                           4  -  D\"{u}ker  Q
                           5  -  h D\"{u}ker
               Sonderprf   6  -  Abflu{\ss} Q
                           7  -  SJerf
                           8  -  Teilf\"{u}llung

                           9  -  Information

      Help : F1                                    Ende : Esc oder F3
______________________________________________________________________

   \end{alltt}
   \normalsize
\end{minipage}
\"{U}ber den Men\"{u}punkt \schalter{8} \afz{Teilf\"{u}llung} wird ein weiteres Untermen\"{u} aufgerufen: \\
\begin{minipage}{1.0\textwidth}
   \footnotesize
   \begin{alltt}


                    \doublebox{\parbox{5.5cm}{        H Y D R A \\ (c.)S O B A U - P S W 1999}}

                         TEILFUELLUNGSWERTE
______________________________________________________________________
                           1  -  Kreis
                           2  -  Normales Ei
                           3  -  Maul DIN 4263
                           4  -  ARMCO (MA,MB,WA,WB)
                           5  -  EA, EB      (ARMCO)
                           6  -  Super-Span  (ARMCO)
                           7  -  REHAU-Sickerrohre

                           9  -  Information

      Help : F1                                    Ende : Esc oder F3
______________________________________________________________________

   \end{alltt}
   \normalsize
\end{minipage}

In den Teilf\"{u}llungsprogrammen werden lediglich geometrische Gr\"{o}{\ss}en von analytisch beschreibbaren Normprofilen bei der
Vorgabe eines F\"{u}llwasserstandes bestimmt. Der \"{U}bergang zur hydraulischen Berechnung erfolgt mit automatischer \"{U}bernahme
der Kennwerte zum Standardprogramm \afz{Hydraulische Leistung von Sonderprofilen}. Die Teilf\"{u}l\-lungsberechnungen setzen
die G\"{u}ltigkeit des Normalabflu{\ss}zustandes voraus (Energieliniengef\"{a}lle = Wasserspiegelgef\"{a}lle = Sohlgef\"{a}lle).


\subsection{Unterprogramm \"{U}berf\"{a}lle}

Im Hauptmen\"{u} steht unter dem Punkt \schalter{3} ein Programm zur Berechnung von \"{U}berf\"{a}llen zur Verf\"{u}gung. F\"{u}r gerade oder
seitlich angestr\"{o}mte Wehre kann f\"{u}r den Fall des vollkommenen Wehres die \"{U}berfallh\"{o}he oder die Abflu{\ss}leistung berechnet
werden. Die Berechnung erfolgt iterativ mit der \"{U}berfallformel nach \autor{Weisbach}. Die \"{U}berfallbeiwerte sind
vorzugeben. \\
\begin{minipage}{1.0\textwidth}
   \footnotesize
   \begin{alltt}


                    \doublebox{\parbox{5.5cm}{        H Y D R A \\ (c.)S O B A U - P S W 1999}}

                      ABFLUSSLEISTUNG \"{U}BERF\"{A}LLE
______________________________________________________________________
                           1  -  QUE   Rechteckwehr
                           2  -  HUE   Rechteckwehr
                           3  -  Streichwehr
                           4  -  Tiroler Wehr
                           5  -  Rundkroniges Wehr
                           6  -  Optimiertes Profil
                           7  -  Dreieckswehr-Thomson

                           9  -  Information

      Help : F1                                    Ende : Esc oder F3
______________________________________________________________________

   \end{alltt}
   \normalsize
\end{minipage}


\subsection{Unterprogramm Wasserspiegel}

Das Untermen\"{u} des Programms Wasserspiegellagen ist wie folgt aufgebaut: \\
\begin{minipage}{1.0\textwidth}
   \footnotesize
   \begin{alltt}


                    \doublebox{\parbox{5.5cm}{        H Y D R A \\ (c.)S O B A U - P S W 1999}}

                         WASSERSPIEGELLAGEN
______________________________________________________________________
                           1  -  Neueingabe
                           2  -  \"{A}nderung
                           3  -  Editor
                           4  -  Rechnen
                           5  -  Querprofil-Plottdatei
                           6  -  L\"{a}ngsschnitt-  "
                           7  -  Plotterausgabe
                           8  -  Graphische Kontrolle
                           9  -  Drucken


      Help : F1                                    Ende : Esc oder F3
______________________________________________________________________

   \end{alltt}
   \normalsize
\end{minipage}
Beim Start des Rechenprogrammes wird zun\"{a}chst eine Steuerdatei \datei{WSP.CTR} mit den Namen f\"{u}r alle Eingabe- und
Ausgabedateien angelegt. Das Rechenprogramm liest diese Steuerdatei ein und legt die Rechenergebnisse unter den in
\datei{WSP.CTR} definierten Datei\-namen ab. Die Steuerdatei \datei{WSP.CTR} enth\"{a}lt:
\begin{itemize}
   \item Pfadname zu den Daten
   \item Eingabedatei  (mit Extension)        \datei{*.WSP}
   \item Ausgabedatei  der Ergebnisse         \datei{*.ERG}
   \item Ausgabedatei  Bewuchsparameter       \datei{*.BEW}
   \item Plottdaten f\"{u}r Querprofile           \datei{*.QPO}
   \item Plottdaten f\"{u}r L\"{a}ngsschnitte         \datei{*.LPO}
   \item Option f\"{u}r gew\"{u}nschten Ausdruck
\end{itemize}
Der Ausdruck von Ergebnissen mu{\ss} anschlie{\ss}end mit Ziffer \schalter{9} des Men\"{u}s bzw. durch Aufruf des Programmes
\datei{DRUCK.EXE} im Hauptmen\"{u} erfolgen. Die Punktbefehle ( vgl. Abschnitt~\ref{Dialogprogramme Subsec Hilfsprogramme} )
werden nur vom Programm \datei{DRUCK.EXE} in Steuer\-be\-fehle f\"{u}r den Drucker \"{u}bersetzt.

Die Dateinamen k\"{o}nnen mit Laufwerksangabe und Extension frei gew\"{a}hlt werden. Dadurch kann beliebig gesteuert werden, wohin
die Ergebnisse geschrieben werden sollen. Ergebnisdateien werden nicht \"{u}berschrieben, f\"{u}r jede Variante wird aus
Sicherheitsgr\"{u}nden ein neuer Dateiname gew\"{a}hlt. F\"{u}r die Benutzung des Dialogprogrammes  zur Eingabe von Wasserspiegeldaten
(nur f\"{u}r DOS-Anwender) ist die Tastenbelegung wie folgt festgelegt:
\begin{quote}
   \begin{tabular}{lp{0.6\linewidth}}
      \schalter{ESC}                               &  Umschaltung in Vormen\"{u} bzw. Abbruch der Dateneingabe \\
      \schalter{Bild $\downarrow$}                 &  Ende der Eingabe in der aktuellen Maske bzw. letzter Punkt \\
      \schalter{Strg} + \schalter{$\leftarrow$}    &  Sprung in Vorfeld \\
      \schalter{Strg} + \schalter{$\rightarrow$}   &  Sprung in Folgefeld \\
      \schalter{Strg} + \schalter{Pos 1}           &  Sprung an den Men\"{u}anfang \\
      \schalter{Strg} + \schalter{Ende}            &  Sprung an das Men\"{u}ende \\
      \schalter{Alt} + \schalter{C}                &  \"{U}bernahme des Vorfeldwertes (Copy)
   \end{tabular}
\end{quote}
Alle \"{u}brigen Cursor-Tasten haben die beim Editieren \"{u}blichen Funktionen.




\clearpage
\subsection{Unterprogramm ATV-Richtlinien}

Der formale Programmablauf und die Druckausgabe der Ergebnisse entsprechen der ATV-Empfehlung zur Berechnung von
Regen\"{u}berl\"{a}ufen~\cite{ATV111}. Zur Eingew\"{o}hnung an den Programmablauf sollten die folgenden Beispiele des Arbeitsblattes
nachvollzogen werden:
\begin{enumerate}
   \item \"{U}berfallwehr mit hochgezogener Schwelle
   \item Spring\"{u}berlauf (f\"{u}r schie{\ss}enden Abflu{\ss})
\end{enumerate}
Das Hauptmen\"{u} ist wie folgt aufgebaut: \\
\begin{minipage}{1.0\textwidth}
   \footnotesize
   \begin{alltt}


                    \doublebox{\parbox{5.5cm}{        H Y D R A \\ (c.)S O B A U - P S W 1999}}

                         ATV - Richtlinien
______________________________________________________________________
                           1  -  Gesamtvolumen A128
                           2  -  Entlastungsbauwerke
                           3  -  RRB und Versickerung
                           4  -  Hilfsprogramme




                           Information

      Help : F1                                    Ende : Esc oder F3
______________________________________________________________________

   \end{alltt}
   \normalsize
\end{minipage}
Mit dem Programm-Modul \afz{HYMATV1} kann das Gesamtspeichervolumen im Einzugsgebiet von Kl\"{a}ranlagen berechnet werden. \\
\begin{minipage}{1.0\textwidth}
   \footnotesize
   \begin{alltt}


                    \doublebox{\parbox{5.5cm}{        H Y D R A \\ (c.)S O B A U - P S W 1999}}

                              HYMATV1
______________________________________________________________________
                           1  -  Gesamtspeicher A128
                           2  -  A128neu Rhl-Pfalz
                           3  -  Pauschalkonzept  Qm
                           4  -  Einzelberechnung Qm
                           5  -  Hilfsprogramme



                           Information


      Help : F1                                   Ende : Esc oder F3
_____________________________________________________________________

   \end{alltt}
   \normalsize
\end{minipage}
Im zweiten Men\"{u}punkt stellt das Programm \afz{ATV-Richtlinien} verschiedene Bemessungsverfahren von Entlastungsbauwerken
nach den ATV-Richtlinien zur Verf\"{u}gung. \\
\begin{minipage}{1.0\textwidth}
   \footnotesize
   \begin{alltt}


                    \doublebox{\parbox{5.5cm}{        H Y D R A \\ (c.)S O B A U - P S W 1999}}

                              HYMATV2
______________________________________________________________________
                           1  -  R\"{U} mit Schwelle
                           2  -  Spring\"{u}berlauf
                           3  -  Fangbecken
                           4  -  Durchlaufbecken
                           5  -  O-Kanalstauraum
                           6  -  U-Kanalstauraum
                           7  -  Hilfsprogramme

                           Information

      Help : F1                                    Ende : Esc oder F3
______________________________________________________________________

   \end{alltt}
   \normalsize
\end{minipage}
Die Bemessung von Regenr\"{u}ckhaltebecken und Versickerungsanlagen ist mit Hilfe des Unterprogramms \afz{HYMAZTV} m\"{o}glich.
\\
\begin{minipage}{1.0\textwidth}
   \footnotesize
   \begin{alltt}


                    \doublebox{\parbox{5.5cm}{        H Y D R A \\ (c.)S O B A U - P S W 1999}}

                               HYMATV3
______________________________________________________________________
                           1  -  Regenr\"{u}ckhaltebecken
                           2  -  Versickeranlagen RAS
                           3  -  Muldensickerung A138
                           4  -  Sickerrigolen   A138

                           6  -  Editor


                           Information

      Help : F1                                    Ende : Esc oder F3
______________________________________________________________________

   \end{alltt}
   \normalsize
\end{minipage}



\clearpage
\subsection{Unterprogramm Kanalnetz}

Das Programm f\"{u}r die Kanalnetzberechnung\footnote{Das Unterprogramm Kanalnetz geh�rt nicht zum eigentlichen Lieferumfang von HYDRAWSP und muss separat erworben werden.} dient zur Berechnung f\"{u}r kleinere Netze ohne R\"{u}ckstaur\"{a}ume und Vermaschungen, bei
denen keine hydrodynamische Berechnung angezeigt ist. Information zu den Grundlagen der Berechnung k\"{o}nnen~\cite{ATV111}
und~\cite{ATV118} entnommen werden. \\
\begin{minipage}{1.0\textwidth}
   \footnotesize
   \begin{alltt}


                    \doublebox{\parbox{5.5cm}{        H Y D R A \\ (c.)S O B A U - P S W 1999}}

                     KANALNETZ  -  ZEITBEIWERT
______________________________________________________________________
                           1  -  Neueingabe
                           2  -  \"{A}nderung
                           3  -  Editor
                           4  -  Rechnen
                           5  -  Drucker - Ausgabe
                           6  -  Psi - Werte


                           9  -  Information

      Help : F1                                    Ende : Esc oder F3
______________________________________________________________________

   \end{alltt}
   \normalsize
\end{minipage}
Der Algorithmus arbeitet nach dem verbesserten Zeitbeiwertverfahren mit Staulinienberechnung gem\"{a}{\ss}~\cite{ATV111}. Die
Berechnung des Reibungsgef\"{a}lles erfolgt nach der Formel von \autor{Prandtl-Colebrook}. \"{O}rtliche Einzelverlustbeiwerte an
den Sch\"{a}chten werden ber\"{u}cksichtigt. Die Ergebnislisten enthalten neben den das Netz beschreibenden Gr\"{o}{\ss}en  (z.B.
Netzzusammenhang, Sammlerl\"{a}ngen, Fl\"{a}chennummern usw.) die anfallenden Regen- und Schmutzwassermengen sowie die Ergebnisse
der hydraulischen Berechnung (z.B. Bemessungswassermengen, Flie{\ss}zeiten, durch Iteration ermittelte Spiegellagen,
Auslastungsgrad).

Das Programm ist besonders gut f\"{u}r die Nachrechnung vorhandener Netze geeignet. Aus den gewonnenen Resultaten kann der
Ingenieur f\"{u}r evtl. erforderliche Sanierungen die Reihenfolge der zweckm\"{a}{\ss}igsten Ma{\ss}nahmen ableiten und durch eine
schrittweise erneute Berechnung die Auswirkung von Teilsanierungen \"{u}berpr\"{u}fen. F\"{u}r Neuplanungen l\"{a}{\ss}t sich durch
schrittweise Verbesserung mit dem Programm eine optimale L\"{o}sung erzielen. Auch der Einsatz f\"{u}r in sehr vielen praktischen
F\"{a}llen vorliegende kombinierte Aufgaben der Neuplanung f\"{u}r Neubaugebiete, die an ein bestehendes Netz anzuschlie{\ss}en sind
ist m\"{o}glich.

In dem Dialogprogramm zur Eingabe der Kanalnetzdaten sind folgende Steuertasten sind fest vereinbart:
\begin{quote}
   \begin{labeling}[]{XXXXXXXXXXX}
      \item [\schalter{F1}]                               Anzeige von Hilfemen\"{u}s bei jeder Tastatureingabe
      \item [\schalter{ESC}]                              Umschaltung in Vormen\"{u} bzw. Abbruch der Dateneingabe
      \item [\schalter{Bild $\downarrow$}]                Ende der Eingabe in der aktuellen Maske mit Abspeicherung der Daten
      \item [\schalter{Strg} + \schalter{$\leftarrow$}]   Sprung in Vorfeld
      \item [\schalter{Strg} + \schalter{$\rightarrow$}]  Sprung in Folgefeld
      \item [\schalter{Strg} + \schalter{Pos 1}]          Sprung an den Zeilenanfang
      \item [\schalter{Strg} + \schalter{Ende}]           Sprung an das Zeilenende
      \item [\schalter{Alt} + \schalter{C}]               \"{U}bernahme des Vorfeldwertes (Copy)
   \end{labeling}
\end{quote}
Alle \"{u}brigen Cursor-Tasten haben die beim Editieren \"{u}blichen Funktionen.

\begin{hinweis}
   Gespeichert wird immer bis zur aktuellen Zeile, von der aus die Eingabe beendet wird. Nachfolgende Zahlen gehen verloren,
   auch wenn sie am Bildschirm noch sichtbar sind!
\end{hinweis}
Nullzeilen am Men\"{u}ende werden nicht \"{u}bernommen. Der Aufruf der einzelnen Masken erfolgt sequentiell von den Startdaten bis
zur Eingabe der stranglosen Fl\"{a}chen (STR3). S\"{a}mtliche Dateien werden in editierbaren ASCII-Files abgelegt.


\subsection{Unterprogramm Stra{\ss}enentw\"{a}sserung}

\begin{minipage}{1.0\textwidth}
   \footnotesize
   \begin{alltt}


                    \doublebox{\parbox{5.5cm}{        H Y D R A \\ (c.)S O B A U - P S W 1999}}

                  ENTW\"{A}SSERUNG VON VERKEHRSFL\"{A}CHEN
______________________________________________________________________
                           1  -  Ablaufabst\"{a}nde
                           2  -  Wasseraufnahme Qz=Qa
                           3  -  Leistung bei \"{U}berstau
                           4  -  Bordrinne/Spitzrinne
                           5  -  Mulde nach RAS-Ew
                           6  -  Rauhbettmulde
                           7  -  Umsetzung P55-LWA
                           8  -  Fl\"{a}chenermittlung

                           9  -  Information


      Help : F1                                    Ende : Esc oder F3
______________________________________________________________________

   \end{alltt}
   \normalsize
\end{minipage}


\clearpage
\subsection{Hilfsprogramme}
\label{Dialogprogramme Subsec Hilfsprogramme}

Hauptanwendung ist die Unterst\"{u}tzung von Betriebssystemfunktionen, ohne das Programm zu verlassen. Mit dem Men\"{u}punkt
\schalter{8} \afz{Printeranpassung} k\"{o}nnen beliebige Dru\discretionary{k-}{k}{ck}er durch Auswahl der aktuellen Datei aus
der Liste der \datei{*.ESC}-Dateien eingestellt werden. Die ausgew\"{a}hlte Datei hei{\ss}t dann \datei{DRUCKER.AKT}. Fehlt der
eigene Drucker in der Auswahlliste, so mu{\ss} eine eigene \datei{*.ESC}-Datei durch Best\"{a}tigung der Abfrage \afz{Aenderung
von Druckersteuerzeichen} hergestellt werden. Im entsprechenden Dialogprogramm sind die ASCII-Zeichen als Dezimalwert oder
explizit (je nach Drucker-Handbuch) f\"{u}r die jeweils gew\"{u}nschten Escape-Sequenzen zu definieren.  In der WSP-Ergebnisdatei
werden z.Zt. nur die Befehle NS / pica / Fettein / Fettaus / Breitein / Breitaus / cond / verwendet. \\
\begin{minipage}{1.0\textwidth}
   \footnotesize
   \begin{alltt}


                    \doublebox{\parbox{5.5cm}{        H Y D R A \\ (c.)S O B A U - P S W 1999}}

                           Hilfsprogramme
______________________________________________________________________
                           1  -  Editieren
                           2  -  Drucken
                           3  -  Auflisten
                           4  -  Kopieren
                           5  -  Umbenennen
                           6  -  L\"{o}schen
                           7  -  Verzeichnis
                           8  -  Printeranpassung

                           9  -  Information

      Help : F1                                    Ende : Esc oder F3
______________________________________________________________________

   \end{alltt}
   \normalsize
\end{minipage}

Das Programm-Modul \schalter{2} \afz{Drucken} besorgt das Ausdrucken von Dokumenten oder sequentiellen ASCII-Dateien. Es
k\"{o}nnen beliebige Namen von Dateien mit Extension eingegeben werden. Die Laufwerksangaben k\"{o}nnen den jeweiligen Dateinamen
vorangestellt werden.
\begin{quote}
   z.B.: \quad \datei{A:KLS.dat} \; oder \; \datei{C:Test.wsp}
\end{quote}
Die f\"{u}r den jeweils angeschlossenen Drucker erforderlichen Steuersignale m\"{u}ssen bei Installation des  Druckers definiert
werden.

%Vorbereitung der zu druckenden Datei
Vom Editor werden z.Zt. nur Schriftarten ohne Attribut generiert. Alle anderen am Bildschirm nicht sichtbaren Schriftarten
m\"{u}ssen nachtr\"{a}glich durch Punktbefehle im Text (am Zeilenanfang oder bei Schriftarten auch an beliebiger Stelle in einer
Zeile) definiert werden. Befehle zur  Drucker-Steuerung, zur Auswahl von Schrifttypen oder von Zeichens\"{a}tzen k\"{o}nnen
ebenfalls in den Text eingef\"{u}gt werden.

Jeder Befehl ist mit einem Punkt abzuschlie{\ss}en. Die Befehle mit  ein/aus  gelten jeweils als Ein- bzw. Ausschalter, z.B:
\begin{quote}
   \texttt{.S FETTein.} \quad  Fettdruck einschalten \\
   \texttt{.S FETTaus.} \quad  Fettdruck ausschalten \\
\end{quote}
Die folgende \"{U}bersicht zeigt eine Auflistung aller unterst\"{u}tzten Punktbefehle:
\begin{itemize}
   \item Befehle zur Druckersteuerung:
      \begin{labeling}[]{XXXXXXXXXXX}
         \item[\texttt{.D init.}]      Drucker-Grundstellung
         \item[\texttt{.D NS.}]        Gehe auf neue Seite
         \item[\texttt{.D SL n.}]      Seitenl\"{a}nge in Zeilen
         \item[\texttt{.D NZ n.}]      Vorschub um n Zeilen
         \item[\texttt{.D PAUSE.}]     Drucker-Halt $>$INFO$<$
      \end{labeling}
   \item Auswahl von Schrifttypen:
      \begin{labeling}[]{XXXXXXXXXXX}
         \item[\texttt{.T pica.}]      10 Zeichen/inch
         \item[\texttt{.T elite.}]     12 Zeichen/inch
         \item[\texttt{.T cond.}]      17 Zeichen/inch
         \item[\texttt{.T PROPein.}]   Proportionalschrift
         \item[\texttt{.T BREITein.}]  Breitschrift
         \item[\texttt{.T ITAein.}]    Kursivschrift
      \end{labeling}
   \item Auswahl von Schriftarten:
      \begin{labeling}[]{XXXXXXXXXXX}
         \item[\texttt{.S FETTein.}]   Fettdruck
         \item[\texttt{.S DOPPein.}]   doppelter Anschlag
         \item[\texttt{.S HOCHein.}]   Hochschreiben ein
         \item[\texttt{.S TIEFein.}]   Tiefschreiben ein
         \item[\texttt{.S UNTein.}]    Unterstreichung
         \item[\texttt{.S NORM.}]      Normalschrift
   \end{labeling}
\end{itemize}
