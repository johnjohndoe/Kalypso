\hypertarget{cmd:datei-speichern}{}
\hypertarget{cmd:datei-speichern-unter}{}
\section{Zeichnungen speichern und exportieren}

Die Plots k�nnen jederzeit �ber \menu{Datei / speichern} gesichert werden. Die Namensvergabe erfolgt dann automatisch. �ber \menu{Datei-speichern unter} haben Sie die M�glichkeit, selbst einen Dateinamen zu vergeben. So k�nnen Sie z.B. auch mehrere unterschiedliche Plots f�r eine Profildatei anlegen, die Sie dann auf die soeben beschriebene Weise (nach vorherigem \menu{Projekt-schliessen} ) wieder �ffnen k�nnen.
Es ist zu empfehlen, die Plots h�ufiger zwischendurch zu speichern, damit aufwendige Formatierungen nicht Gefahr laufen, verloren zu gehen. 

\hypertarget{cmd:datei-dxf-export}{}
Alternativ k�nnen Sie �ber \menu{Datei / Dxf Export} eine dxf-Datei erstellen, die Sie in einem CAD-Programm
einlesen und ggf. weiter aufbereiten k�nnen.