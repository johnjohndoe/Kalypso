\hypertarget{cmd:multiplot}{}
\section{Mehrere Plots auf ein Blatt}
\label{sec:multiplot}

Wird beim Laden der Profildaten die Option \ctrl{Mehrere Plots auf ein Blatt} selektiert (siehe Abbildung~\ref{fig:auswahl-profile}), so werden alle ausgew�hlten Profile in einem Plotter-Fenster ge�ffnet. Die Profile werden dabei mit der vorgegebenen Standardvorlage (siehe Abschnitt~\ref{sec:templates}) formatiert und Zeilenweise angeordnet. Ist in der Standardvorlage ein Stempel ausgew�hlt, so wird dieser rechts unten plaziert.

\begin{figure}[htbp]
	\centering
	% FEHLT!
	%\includegraphics[width=1.0\textwidth]{multiplot} 
	\caption{Mehrere Plots auf einem Blatt}
	\label{fig:multiplot}
\end{figure}

Passen nicht alle Profile auf ein einzelnes Blatt (zum Einstellen der Seitengr�sse siehe Abschnitt~\ref{sec:print_plot}), so werden wie im Plotter �blich mehrere Bl�tter erstellt und die Blattgrenzen durch blaue Linien dargestellt. Passt ein Profil nicht vollst�ndig auf ein Blatt, so wird es komplett weggelassen und eine entsprechende Meldung ausgegeben.

Prinzipiell stehen im Modus 'mehrere Plots auf ein Blatt' die gleichen Bearbeitungsm�glichkeiten zur Verf�gung wie im normalen Modus. Lediglich die Profildarstellung wird ausschliesslich �ber die Standardvorlage gesteuert und kann nachtr�glich nicht mehr ver�ndert werden. Statt dem entsprechenden Dialog im Einzelprofilmodus steht durch das Men� \menu{Bearbeiten / Eigenschaften} ein Dialog zum Einstellen der Abst�nde zwischen den Profilen und zum �ndern der Blatt�berschrift zur Verf�gung.

\begin{figure}[htbp]
	\centering
	% FEHLT
	%\includegraphics[width=0.6\textwidth]{multiplot_props}
	\caption{Einstellungen f�r 'Mehrere Plots auf ein Blatt'}
	\label{fig:multiplot_properties}
\end{figure}