\section{Einf�hrung}
Das Programm \wspmap\ dient der Erstellung und Visualisierung von Projekten zur eindimensionalen Wasserspiegellagenberechnung mit dem Programm \wspwin.\\
Das vorliegende Programm erm�glicht die Ein- und Ausgabe der im Programm  \wspwin\ verwendetens hydraulischen Daten in Form von Lagepl�nen (Karten).\\
Aus dem mit \wspwin\ berechneten Wasserspiegellagen k�nnen mit Hilfe eines Digitalen Gel�ndemodells (DGM) �berschwemmungsfl�chen und neuerdings auch Flie�tiefen generirt werden.\\
Ferner bietet \wspmap\ die M�glichkeit, vorhandene Querprofile weiter zu beachten (Verl�ngerung der Vorl�nder) bzw. zus�tzliche Querprofile in ein Projekt einzuf�gen. Durch die Import - Funktion von Vermessungsdatens�tzen ist es m�glich, Projekte neu zu erstellen und anschlie�end die Berechnung mit \wspwin\ vorzunehmen.\\
Durch die Einbindung von ESRI Map - Opjects 2 entspricht \wspmap\ der Konzeption der ESRI - Produkten Aice Vier und ArcExplorer.\\
Beim Einlesen der \wspwin\ - Datens�tze werden somit automatisch ESRI - konpatible Daten (Shape - Files) erzeugt. Entsprechend lassen sich externe Daten in \wspmap\ visualisieren und teilweise bearbeiten.\\\\
Im einzelnen beinhaltet \wspmap\ die folgenden Funktionen:
 
\begin{itemize}
	\item 
	Visalisierung
	
		\begin{itemize}
			\item
			Darstellung von georeferenzierten \wspwin\ - Querrprofilen, inklusiveder Flie�zonen (Trennfl�chen, Bordrollpunkten, Durchstr�m Bereichen) im Lageplan.
			\item
			Hinzuladen von weiteren Informationen, wie Hintergrundkarten, Luftbilder oder beliebigen Shape-Files / dxf - Files
			\item
			Import von Vermessungsdaten (da66, da65, tripple) und Erzeugen von \wspwin\ - Datens�tzen
		\end{itemize}
		
	\item 
	Datenbearbeitung im Lageplan
	
		\begin{itemize}
			\item 
			Verschieben der Flie�zonen (Trennfl�chen, Durchstr�mbereiche usw.)
			\item
			Verl�ngern der Profile (Vorlandbereiche) mit Hilfe eines digitalen Gel�ndemodells (DGM)
			\item
			Erzeugen zus�tzlicher Querprofile (DGM notwendig)
			\item 
			Zuweisung von Rauheiten �ber Nutzungsklassen 
			\item
			Speichern der �nderungen und Weiterverarbeitung  (Berechnung) in \wspwin .							\end{itemize} 		
			
	\item
	Weiterverabeitung  von Berechnugsergebnissen
	
		\begin{itemize}
			\item 
			Darstellung berechneter Wasserspiegellagen (als Punkte im Querprofil)
			\item
			Erzeugen von �berschwemmungsfl�chen durch Verschiedene Wasserspiegel mit einem DGM
			\item
			Ermittlung und Darstellen von Flie�tiefen (Verechnung von Wasserspiegellagen mit DGM)
			
			$\Longrightarrow$ Ausgabe (Plot) von �berschwemmungs- und Flie�tiefenkarten
		\end{itemize}
		
\end{itemize}
	