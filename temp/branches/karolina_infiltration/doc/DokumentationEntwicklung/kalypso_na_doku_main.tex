\documentclass[12pt,titlepage,a4paper,leqno]{report}
\def\LATEX{\LaTeX}  \let\TEX = \TeX

\special{!userdict begin /setpagedevice { pop } bind def end}

\usepackage[ngerman]{babel}         % Laden von Deutschen Eigenschaften, neue Rechtschreibung

\usepackage[T1]{fontenc}            % Aktiviert das T1 Encoding
\usepackage[latin1]{inputenc}       % Latin1 - Kodierung bei Tastatureingabe, erm�glicht �,� usw.

\usepackage{graphicx}               % Grafikbefehle einbinden

\usepackage[dvips]{epsfig}

\usepackage{fancyhdr}               % Kopf- und Fu�zeile
\usepackage{rotating}               % Erlaubt das Rotieren von eps-Bildern und Tabellen
\usepackage{tabularx}               % Erweiterte Tabellen-Befehle
\usepackage{longtable}              % Erlaubt verwendung von longtable
%\usepackage{multirow}               % Mehrzeilige Spalten
\usepackage{amsmath}                % Verbesserte M�glichkeiten f�r den Mathe-Modus
\usepackage{amssymb}                % Erweiterte Mathematische Symbole
\usepackage{exscale}                % Deaktivierung der cmex10 Schw�che f�r den Mathe-Modus
\usepackage{textcomp}               % Darstellung der TC-Fonts
\usepackage{verbatim}               % Erweiterung der verbatim-Umgebung
\usepackage{mathptmx}               % Schriftart Times mit angepasster Mathe-Schrift
\usepackage{pifont}                 % Bereitstellung von PostScript Sonderzeichen
\usepackage{hyperref}               % Automatische Verlinkung im Dokument
\usepackage{eurosym}                % Bereitstellung des \euro - Befehls
\usepackage{bibgerm}
\usepackage{wrapfig}
\usepackage{makeidx}                % This contains the macros for indexing
\usepackage{float}
\usepackage{hhline}                 % Erm�glicht weitere Linientypen in der Tabellenumgebung
\usepackage{color}
%\usepackage{listings}


%\renewcommand{\familydefault}{\sfdefault} % Umschalten aus Serifenlose Schrift (Arial)

%Seitenr�nder festlegen: 2cm oben/unten, 2cm links/rechts
\setlength{\oddsidemargin}{0cm} \setlength{\topmargin}{-1,5cm}
\setlength{\headsep}{1,2cm} \setlength{\textwidth}{16,5cm}
\setlength{\textheight}{23,5cm}
%\setlength{\mathindent}{2em}
\setlength{\footskip}{15mm} \setlength{\parindent}{0em}
\setlength{\headheight}{28.3pt}
%Seitenr�nder fertig

\renewcommand{\textfraction}{0}

%
%************ K O P F Z E I L E ********************************************************************************************************************************************
%

\pagestyle{fancy} \lhead{\sc Arbeitsbereich Wasserbau\\ \sc TU Hamburg-Harburg}
%\rhead{\includegraphics*[height=8mm]{bilder_allg/wb_logo4c}} %\cfoot{\thepage}
\setlength{\parskip}{1.5ex}
\renewcommand{\baselinestretch}{1.5}
\normalsize

%\psdraft

%
%*******************EIGENE BEFEHLE**************************************************************************************************************************************
%
\newcommand{\bl}{\symbol{'040}}     %das Symbol f�r ein Blank wird durch \bl erzeugt




%
%************ D O K U M E N T ********************************************************************************************************************************************
%

%\makeindex
\begin{document}

\begin{titlepage}
\thispagestyle{empty}
\begin{center}
\begin{tabular}{l@{\hspace{40mm}}r}
\includegraphics*[height=13mm]{bilder_allg/tu_logo_farbe} &
\includegraphics*[height=13mm]{bilder_allg/wb_logo4c}
\end{tabular}

\vspace{3cm}
{\Huge Dokumentation zur Softwareentwicklung}\\
\vspace{3.5cm}


{\huge Kalypso-NA Fortran}\\

\vspace{3.5cm}
%\begin{center}
%\includegraphics*[height=8cm]{bilder_allg/titelbild_neu}
%\end{center}
\vspace{5.5cm}
Dipl.Ing. Jessica Nordmeier\\
\vspace{2cm}


Hamburg, \today

\end{center}
\end{titlepage}



%
%************ L I S T E N ********************************************************************************************************************************************
%
\thispagestyle{empty}

\pagenumbering{Roman}

%\tableofcontents \pagebreak \listoffigures \pagebreak \listoftables \pagebreak
\newpage \pagenumbering{arabic}

%
%*********** I N P U T - D A T E I N *********************************************************************************************************************************
%


\newpage
%\input{kalypso_java}
\chapter{Kalypso-NA Fortran}
�bersicht der Namensk�rzel:

\begin{tabular}{|ll|}
\hline
  CB: & Claudia Br�ning \\
  EP: & Erik Pasche \\
  FN: & Frank Nohme \\
  JH: & Jessica H�bsch \\
  JN: & Jessica Nordmeier\\
  RS: & Rolf Schr�der\\
  SK: & Sephan Kr��ig\\
  AvD:& Andreas von D�mming\\
  CK: & Christoph K�pferle\\
  KV: & Karolina Villagra-Mendoza\\
\hline
\end{tabular}


\section{Versionen}
\begin{table}[H]
  \centering
  \begin{tabular}{|p{2.5cm}|c|c|p{8.5cm}|}
    \hline
    Version       & von        & bis    & neu in dieser Version \\
                  & Release (cvs-tag)    & & \\
    \hline
    Version 00    &            &                                &  \\
    Version 01    &            &                                &  \\
    Version 02    &            &                                &  \\
    Version 03    &            & 08.07.2004                     &  \\
    Version 2.0   & 08.07.2004 & 12.07.2004                     & Es wird ausschlie�lich das grap-Format als
                                                                  Niederschlagsformat akzeptiert.\\
    Version 2.0.1 (Ingb�ros) & 12.07.2004 & 12.08.2004          & Durch Analyzer Programm aufger�umt (Fortran Standard,
                                                                  Arguments, Variablen), nicht verwendete Routinen
                                                                  gel�scht.\\
    Version 2.0.2 & 12.08.2004 &                                & Anfangswerte auch aus der Kurzzeitsimulation schreiben.
                                                                  Faltung des Interflows mit ZFT anstelle des
                                                                  Parallelspeichers bei Berechnung mit Hydrotopen. Fehler in
                                                                  der Berechnung des CIN- und CEX-Wertes behoben.
                                                                  Format�nderungen: Darstellung Bodenfeuchte im GraphicTool,
                                                                  Format der Ausgabendateien mit Anfangsdatum\\
    Version 2.0.3 & 2005       &  01.04.2006                           &\\
    Version 2.0.4 & 01.04.2006 &  19.04.2006                    &Aufteilungfaktor Fluss/ Vorland pro Abfluss \\
    Version 2.0.5 & 19.04.2006 &  09.08.2006                    & Mulden Rigolen Berechnung, Format Hydrotopdate\\
    Version 2.0.6 & 09.08.2006 &  13.06.2007                    & Logger, KM-Berechnung\\
    Version 2.0.7 & 13.06.2007 &                                & van Genuchten und Makroporen\\
    Version 2.0.8 &  &                                & Umstellung auf g95 Compiler\\
    \hline
  \end{tabular}
  \caption{Versions�bersicht}
\end{table}

Auf alte �nderungen kann mit Hilfe der �nderungsnummer Bezug genommen werden (z.B. V03m1 - �nderung von Version 02 zu
Version 03 Nummer 1).

\subsection{Version 01 zu Version 02}

\textbf{Erh�hung der Anzahl m�glicher Zeitschritte in der KZsimulation (CB, 26.10.2002) (V02m1)}\\
In param.cmn idimt, idimtgw und idim von 1086 auf 2880 gesetzt.

\textbf{Initialisierung der Variable m\_bmax in der Datei inp\_bod\_b.f90 (FN, 29.10.2002) (V02m2)}\\
Variable/Feld m\_bmax mit 0.0 initialisiert. Grund: Wenn Bodenarten nicht richtig aus der Hydrotop-Datei eingelesen
werden, kommt es jetzt direkt zu einer Fehlermeldung. Vorher wurde die do-Schleife (Zeile 109 bis 185) nicht
durchlaufen, da m\_bmax nicht mit Werten ungleich Null "aufgef�llt" wurde. Es kam erst im Anschluss zu der
Fehlermeldung (Zeile 243), welche nicht eindeutig auf die Fehler in der Hydrotop-Datei hinwies.

\textbf{Jahr 2000 F�higkeit (SK/EP, 10.03.2003) (V02m3)}\\
Gerechnet werden kann nur eine Langzeitsimulation < 100 Jahre (max. 99 Jahre), da das Jahr zum Teil noch 2stellig
�bergeben wird. Ansonsten vollst�ndige Jahr 2000 F�higkeit f�r Kurz- und Langzeitsimulation hergestellt.
\begin{itemize}
    \item Anfangswerte:\\
          Anfangswerte in der *.konfig Datei sind 8-stellig anzugeben: jjjj,mm,tt\\
          Sollen zu jedem Zeitschritt der Langzeitsimulation Anfangswerte geschrieben werden, ist in der *.konfig Datei als
          Anfangswert f�r die Kurzzeitsimulation nur 1000001 anzugeben. Werden in der Langzeitsimulation f�r jeden Tag die
          Anfangswerte herausgeschrieben, kann der erzeugte lzsim Ordner ausgelagert und in der *.konfig Datei der entsprechende
          Pfad angegeben werden mit lzpath=....  Liegt der lzsim Ordner z.B. unter c:$\backslash$daten$\backslash$lzsim ist
          lzpath=c:$\backslash$daten anzugeben. Der Pfad wird nach der beendenden 99999 nach den Anfangswerten angegeben.
          (Feature f�r KalypsoRRM und Kalypso Forecast Tool).
    \item Eingabeformate:
    \begin{itemize}
        \item Niederschl�ge:\\
              Langzeit\\
              Jahresblock nasim\\
              1.Zeile Kommentar\\
              2.Zeile 7x,jjjj,mm,tt...\\
              \\
              Kurzeit (grap-Format)\\
              1. Zeile Kommentar\\
              2. Zeile 7x,jjjj,mm,tt...
        \item Klimadaten: (TV-Datei) keine �nderung zum alten Format\\
    \end{itemize}
\end{itemize}

\subsection{Version 02 zu Version 03}

\textbf{Formate Eingabedatei Langzeitniederschl�ge ge�ndert (SK, 25.07.2003) (V03m1)}\\
Das Eingabeformat der Langzeitniederschl�ge wird ge�ndert um die Kommunikation mit dem Datacenter zu erleichtern. Die
Niederschl�ge werden jetzt mit dem Grap-Kurzzeitformat eingelesen. Das Nasim-Blockformat wird nicht mehr unterst�tzt.
Der Inhalt der Subroutine input\_nied\_lz wird durch die Einleseroutine input\_nied\_kz, die die Datei
input\_nied\_kz\_dat aufruft, ersetzt.

$\Rightarrow$ Kommentar: JH, 16.06.2004\\
In der inp\_nied\_kz wird die Subroutine n\_approx aufgerufen, welche den Niederschlag auf das Simulationszeitintervall
approximiert. Dies ist bei einer Langzeitsimulation nicht erforderlich, da das Niederschlagsintervall dem
Simulationsintervall entspricht (24h).\\
Au�erdem wird in der inp\_nied\_kz der eingelesene Niederschlag in eine Niederschlagsintensit�t umgerechnet. Dies
geschieht jedoch nach dem Aufruf der inp\_nied\_lz (welche durch die inp\_nied\_kz ersetzt ist) aus der Subroutine
input nochmals und ist daher �berfl�ssig. Es geschieht jedoch kein Fehler, da bei der Umrechnung in der inp\_nied\_lz
die Umrechnungsparameter faktn und dt beide zu 24 gesetzt sind und sich damit ein Faktor von 1 ergibt. Siehe auch
V2.0m1, V2.0m2.

\textbf{Anfangsstunde Langzeitsimulation wird auf 00 gesetzt (SK, 25.07.2003) (V03m2)}\\
Um den Austausch mit dem Datacenter zu harmonisieren wird die Anfangsstunde f�r die Langzeitsimulation auf 00 Uhr
gesetzt (vorher 7:30). Dies geschieht in der Routine inp\_nied\_kz\_dat.dat (Bei Kurzzeitsimulation wird nach wie vor
die Stunde aus der falstart.lst eingelesen).


\textbf{Niederschlagsverschiebung um ein Niederschlagsintervall zur�ck (EP, 07.11.2003) (V03m3)}\\
Bei der Langzeitsimulation ist eine Verschiebung der Niederschl�ge um einen Tag zur�ck erforderlich, da der DWD die
Tagesniederschl�ge immer f�r den Zeitraum von 7.30 Uhr eines Tages bis 7.30 Uhr des n�chsten Tages misst. D.h. der in
der GRAP-Datei am Tag X angegebene Niederschlag bezieht sich immer auf den Tag davor.

$\Rightarrow$ Kommentar: JH, 14.06.2004\\
Die �nderung erfolgt durch das Zur�cksetzen des eingelesenen Niederschlages um ein Niederschlagsintervall. Da die
Ausf�hrung in der inp\_nied\_kz\_dat geschieht, gilt sie nach �nderung V03m1 f�r Kurz- und Langzeitsimulation.

\textbf{Definition der Korekturfaktoren Interflow (EP, 02.11.2003) (V03m4)}\\
Die Definition von m\_retlay in der Subroutine gebiet sind unlogisch, da m\_retlay bereits in der Subroutine
inp\_bod\_b definiert wird. Daher werden diese Zeilen auskommentiert. Ist der zusammengesetzte Korrekturfaktor
m\_retlay ($m\_retlay = xret \cdot retlay$) gr��er als 1, so wird dieser zu 1 gesetzt (Zeile if (m\_retlay(ilay,nb)
.gt. 1.) m\_retlay(ilay,nb) = 1. war zuvor versehentlich auskommentiert worden).

Wegen doppelter Definition Korrektur Interflow:\\
1: in Eingabedatei Bodentyp (boden.dat): xret (wird von Routine inp\_bod\_b gelesen), sollte nur 0 (kein Interflow in
Bodenschicht zul�ssig) oder 1 Interflow in Bodenschicht zul�ssig) sein.\\
2: in Eingabedatei Gebiet (*.geb): m\_retlay (wird von Routine gebiet gelesen). Diese ruft wiederum die Datei
inp\_bod\_b   auf. Beide Korrekturfaktoren werden multipliziert.\\
Problem: Korrekturfaktor-Interflow retlay darf nie gr��er 1 werden, da dieser Faktor multiplikativ auf das
Interflowvolumen wirkt. Die  Wasserverluste k�nnen dadurch so gro� werden, dass die Bodenfeuchte negativ wird. Dann
wird in der Kurzzeitsimulation w�hrend der Abarbeitung jedes Hydrotops, jeder Bodenschicht eine Fehlermeldung
geschrieben. Daher in Bodendatei als Null oder 1 definieren und in Gebietsdatei niemals gr��er 1 setzen!

\textbf{Anzahl der Niederschlagswerte darf die Anzahl der Simulationszeitschritte �berschreiten (???) (V03m6)}\\
In der Routine input wird die Sicherheitsabfrage so ge�ndert, dass die Anzahl der Niederschlagswerte gr��er sein darf
als der Simulationszeitraum, aber nicht umgekehrt.\\
Vorher: (idif.ne.izaehl.or.ierr.gt.0)\\
Jetzt: (idif.gt.izaehl.or.ierr.gt.0)

\textbf{Einlesen der Niederschlagsstation f�r jedes Teilgebiet m�glich (???) (V03m7)}\\
Es ist m�glich, f�r jedes Teilgebiet die verwendete Niederschlagsstation anzeigen zu lassen. Daf�r wurde Abfrage in
Gebietsdatei auskommentiert, die pr�fte, ob f�r Station schon in einem anderen Teilgebiet vorgekommen ist.

\textbf{M�gliche Formate Eingabedatei Temperatur Verdunstung erweitert () (V03m8)}\\
Um eine bessere Komunikation mit dem DataCenter zu erm�glichen, ist es jetzt auch m�glich die Zeitreihen f�r Temperatur
und Verdunstung in getrennten Dateien jeweils nach dem Ex2-Format einzulesen:\\
EX2-Format mit tt:mm:jjjj ss:mm Wert, Format(2(i2,1x),i4,1x,i2,1x,a)\\
Dies geschieht in der input\_nied.tv-Datei. Diese Datei hie� vorher input\_nied.lz. Die Aufrufe der Subroutine zum
Einlesen der Temperatur- und Verdunstung wurde von input\_nied.lz in inp\_nied.tv umbenannt. Die Eingabedateien m�ssen
aus der Gebietsdatei wie folgt aufgerufen werden:\\
xxxxx.tmp:yyyyy.verd (gleiche Zeile). Es ist auch noch m�glich das alte TV-Format (Temperatur und Verdunstung in einer
Datei) zu verwenden.

\textbf{�nderungen um maximale Anzahl von Bodentypen und Landnutzungen zu erh�hen (SK, 21.10.2003) (V03m9)}\\
In Datei param.cmn wurden folgende Variablen ge�ndert:\\
Idimnutz = 5000\\
Idimnutz2 = 100\\
Idimnutz3 = 100\\
In der Routine inp\_bod\_020.f90 wird die Abfrage nach der Bodentypenanzahl if(ityp .gt 50) auf if(ityp .gt idimnutz2)
ge�ndert. Auch in der Variablendeklaration am Beginn der Subroutine, bei der Deklaration der Variablen f�r die
Bodenkennwerte, wird 50 auf idimnutz2 gesetzt.

\textbf{�nderung Synthetische Niederschl�ge (SK) (V03m10)}\\
Es soll erm�glicht werden, dass f�r jedes Teilgebiet eine Datei mit synthetischen Niederschl�gen eingelesen werden
kann. In der Version 02 wurde der Name der Gebietsdatei aus der falstart.lst gelesen. Der Leseaufruf erfolgte aus der
Hauptroutine (call checkn). Bei Erreichen des Aufrufs call gebiet wurden dann keine unterschiedlichen
Niederschlagsdateinamen f�r einzelne Teilgebiete eingelesen. �nderungen in folgenden Routinen sind daher in Version 03
vorgenommen worden:
\begin{itemize}
    \item Routine bcena3 (sp�ter ge�nderte Bezeichnung: kalypso-na)
    \begin{itemize}
        \item Call checkn wurde auskommentiert.
        \item Bei dem Aufruf call Gebiet wurden zwei Parameter xjah und xwahl2 am Ende �bergeben.
        \item Die Parameter pdsol (Dauer des Niederschlags) und pjsol (J�hrlichkeit) werden durch die Parameter xjah und
              xwahl2 ersetzt.
    \end{itemize}
    \item Routine gebiet
    \begin{itemize}
        \item xjah und xwahl2 wurden als Real deklariert
        \item beim Aufruf call input werden die Parameter xjah und xwahl2 �bergeben
    \end{itemize}
    \item Routine input
    \begin{itemize}
        \item xjah und xwahl2 wurden als Real deklariert
        \item an Marke 32 wird die Routine checkn aufgerufen, die in der Gebietsdatei auskommentiert wurde. Call checkn
    \end{itemize}
    \item Routine checkn wurde ge�ndert
\end{itemize}

\textbf{Hydrotopfl�che und Versiegelungsgrad wird nur aus der Hydrotopdatei �bernommen (SK/CB) (V03m11)}\\
In Version 02 m�ssen Versiegelungsgrad und Hydrotopfl�che in Gebietsdatei und Hydrotopdatei eingegeben werden. Zwischen
beiden erfolgte ein Abgleich. Um bei �nderungen in den Hydrotopfl�chen �ber die Kalypsooberfl�che keine Redundanzen
ber�cksichtigen zu m�ssen, wird die Hydrotopfl�che jetzt immer aus der Hydrotopdatei �bernommen (siehe Routine gebiet)
und automatisch �ber das Objektmodell in die Gebietsdatei �bertragen.

\textbf{Routine inp\_hydro Fehlermeldungen (EP) (V03m12)}\\
Fehlermeldungen  optimiert


\textbf{Herausgefundene Fehler durch Eingabedateien (EP) (V03m13)}
\begin{itemize}
    \item Aufteilungsfaktor Grundwasser/Tiefengrundwasser wird an zwei Stellen gesetzt. Es d�rfen niemals beide ungleich
          1 gesetzt werden, da dann Fehler in Boden-Routine, die die Grundwasserroutine aufruft, auftritt.
    \begin{enumerate}
        \item Hydrotopdatei (Variable: m\_f1gws)
        \item Gebietsdatei (Variable: rtr)
    \end{enumerate}
    \item Korrekturfaktor banf in Gebietsdatei wird bei fehlendem Ordner lzsim als Startwert eingelesen
\end{itemize}

\textbf{Unit Hydrograph, Routine isov (EP, 05.06.03) (V03m13)}\\
Wenn  Retentionskonstanten sehr viel gr��er als der Simulationszeitraum sind, kann der Unit Hydrograph innerhalb des
Simulationszeitraums nicht abgearbeitet werden. Um dies zu vermeiden, wird der Unit Hydrograph in diesem Fall dann so
korrigiert, dass er auf alle innerhalb des Simulationszeitraums komplett abgearbeitet werden kann.\\
UH wird ung�nstig ver�ndert, wenn sehr hohe Retentionskonstanten. Hierdurch kann das Ergebnis von isov sehr stark von
der Dauer der Simulation (idif) abh�ngig werden. idif+ituh immer gleich idim. Das bedeutet, wenn idif klein ist, wird
ituh gro� und umgekehrt. Bei gro�er Retentionskonstante wird der UH nicht vollst�ndig im Zeitraum von ituh liegen, d.h.
der UH wird mit einem Korrekturfaktor multipliziert, so da� nach der Korrektur der UH auf den Wert 1.0 normiert ist.
Wenn ituh kurz ausf�llt, aufgrund eines langen Simulationszeitraumes, dann wird die Korrektur gr��er ausfallen, als bei
kleinem Simulationszeitraum. Hierdurch entsteht eine gr��ere Stauchung des UH bei langem Simulationszeitraum. Dies
f�hrt zu dem paradoxen Ergebnis, da� bei langem Simulationszeitraum die durch isov gefaltete Ganglinie schneller
abl�uft als bei kurzem Simulationszeitraum. Zur Vermeidung dieses Paradoxom wird ituh= idim gesetzt (�nderung von RS
wird r�ckg�ngig gemacht). In diesem Fall bleibt bei hohen Retentionskonstanten immer noch ein Einflu� der
Simulationszeit bestehen, allerdings wird der Widerspruch vermieden.

\subsection{Version 03 zu Version 2.0}
\textbf{Subroutine inp\_nied\_lz ersetzt (JH, 22.06.2004) (V2.0m1)}\\
Da die �bernahme der inp\_nied\_kz in die inp\_nied\_lz zu verschiedensten Problemen f�hrte (Verschiebung der
Niederschl�ge Ein-/Ausgabe, Approximation (V03m1 Kommentar)), wurde die\\
inp\_nied\_lz �berarbeitet und inp\_nied\_kz und inp\_nied\_kz\_dat zu einer neuen Subroutine\\
inp\_nied\_lz zusammengef�gt. Es wird jetzt lediglich das grap-Format der Niederschlagsdatei Langzeit akzeptiert.\\
Die Niederschlagsdaten bleiben nach V03m3 um einen Tag zur�ck verschoben. Au�erdem wurde der Fehler bei der Ausgabe der
Anzahl eingelesener Niederschlagswerte behoben (zuvor wurden hier unsinnige Angaben gemacht 369 o.�.).

\textbf{Subroutine inp\_nied\_kz\_dat zur�ckgesetzt (JH, 22.06.2004) (V2.0m2)}\\
Da es eine eigenst�ndige Subroutine zum Einlesen der Langzeitniederschl�ge im grap-Format gibt (inp\_nied\_lz, V2.0m1),
sind die Spezifikationen zum Langzeitniederschlag in der Subroutine inp\_nied\_kz\_dat nicht mehr notwendig. Aus diesem
Grund k�nnen die �nderungen V03m2 und V03m3 f�r die Kurzzeitsimulation in inp\_nied\_kz\_dat r�ckg�ngig gemacht werden.
Da Ausschlie�lich das grap-Format verwendet werden soll, werden die anderen Formate gel�scht, wobei beim Aufruf dieser
Formate durch den Benutzer eine Fehlermeldung erscheint.

$\Rightarrow$ Kommentar: JH, 28.06.2004\\
Es bleibt nicht gekl�rt, ob die Kurzzeitniederschl�ge je korrekt eingelesen wurden. Dies ist zu pr�fen! Au�erdem muss
�berpr�ft werden in welcher Form die Kurzzeitniederschl�ge geschrieben werden, da es m�glicherweise auch hier sinnvoll
ist die Werte um eine Intervall zur�ckzusetzten.

\subsection{Version 2.0 zu Version 2.0.1}
\textbf{Aufr�umen (JH, 12.07.2004) (V2.0.1m1)}\\
Der Code wurde durch eine Analyse im Analyzer aufger�umt. Es sind hierbei die �berfl�ssigen Routinen outnas,
outarcview\_tg, qbm\_hws, isovtgw (durch isov ersetzt) und hydographt (durch hydograph ersetzt) gel�scht worden (im cvs
noch vorhanden bis zum 08.07.2004). Au�erdem wurde der Fortran Standard und die Argumente mittels der Funktion Check
gepr�ft und verbessert/hergestellt. Die bei der Analyse auftretenden Hinweise zu nicht verwendeten Variablen wurden
(soweit dies m�glich war) ebenfalls ber�cksichtigt und die entsprechenden Variablen gel�scht. Zus�tzlich wurde eine
Umbenennung der Dateien vorgenommen (Versionsbezeichnung 03 entfernt).

\textbf{Anzahl der Anfangsbedingungen auf 100 erweitert (JH, 19.07.2004) (V2.0.1m2)}\\
Die Anzahl der m�glichen Anfangsbedingungen in der *.konfig wurde auf 100 erweitert (vorher 15). Hierzu wurde der
Parameter idimanf im Common-Block param.cmn auf 100 gesetzt. Au�erdem wurden in der Routine kalypso-na die
Fehlermeldungen bez�glich der maximalen Anzahl an Anfangsbedingungen optimiert.

\textbf{Verwirrung bez�glich bcena Benennungen aufgehoben (JH, 09.08.2004) (V2.0.1m3)}\\
Alle bcena Bezeichnungen wurden entfernt, bzw. durch KALYPSO-NA ersetzt. Au�erdem wird beim Berechnungsstart ein
Hinweis zum Modell (Version, Autor...) ausgegeben (Bildschirm und output Dateien).

\subsection{Version 2.0.1 zu Version 2.0.2}
\textbf{Anfangswerte auch aus der Kurzzeitsimulation schreiben (JH, 24.-30.09.2004) (V2.0.2m1)}\\
Es k�nnen jetzt auch Anfangswerte aus einer Kurzzeitsimulation neu erzeugt werden. Hierbei waren folgende �nderungen
notwendig:
\begin{enumerate}
    \item Einlesevorgang aus *.konfig\\
          Die Daten der zu schreibenden Anfangswerte in der *.konfig wurden um die Angabe der Stunde erweitert. Format:
          yyyymmdd hh (i8,2x,i2). Falls diese nicht angegeben werden, so werden sie zu Null gesetzt. �nderung erfolgt
          beim Einlesen der Werte aus der *.konfig, kalypso-na.f90 Z. 1422 ff.
    \item Anfangswerte schreiben *.lzg\\
          Die Anfangsdaten werden in der Langzeitsimulation wie gewohnt geschrieben, jedoch hat sich das Format durch
          den Zusatz der Stunden bei der Datumsangabe ge�ndert. In der Kurzzeitsimulation wird die *.lzg Datei jetzt
          auch geschrieben. Neues einheitliches Format der *.lzg-Datei:\\
          nanf(i1),nanfstd(i1),' h','    1',' qgs' \hspace{2cm} (i8,2x,i2,a2,a4,a4)')\\
          '1',qz(i) \hspace{2cm} (a4,f9.3)\\
          Die Variable nanfstd ist neu und stellt die Zeit in Stunden dar, zu der die jeweiligen Anfangswerte
          geschrieben werden sollen. In der Langzeitsimulation werden die Werte automatisch durch '00' ersetzt.
          �nderungen in kalypso-na.f90 Z. 1943 ff.
    \item Anfangswerte schreiben *.lzs\\
          Die Anfangsdaten werden in der Langzeitsimulation wie gewohnt geschrieben, jedoch hat sich das Format durch
          den Zusatz der Stunden bei der Datumsangabe ge�ndert. In der Kurzzeitsimulation wird die *.lzs Datei jetzt
          auch geschrieben. Neues einheitliches Format der *.lzs-Datei:\\
          Grundwasser:\\
          nanf(ianf),nanfstd(ianf),' h','   1','gwsp' \hspace{2cm} (i8,2x,i2,a2,2x,a4,1x,a4)\\
          1,hgws(t),qb(t) \hspace{2cm} (i4,f9.2,f9.3)\\
          Boden:\\
          nanf(ianf),nanfstd(ianf),' h',anzelem,'bodf' \hspace{2cm} (i8,2x,i2,a2,2x,i4,1x,a4)\\
          nn,bi(nn),(bof(nn,ilay),ilay=1,anzlayy) \hspace{2cm} (i4,f7.2,10(f7.2))\\
          Schnee:\\
          nanf(ianf),nanfstd(ianf),' h','   1','snow' \hspace{2cm} (i8,2x,i2,a2,2x,a4,1x,a4)\\
          1,h(nt),ws(nt) \hspace{2cm} (i4,2(f9.2))\\
          Die Variable nanfstd ist neu und stellt die Zeit in Stunden dar, zu der die jeweiligen Anfangswerte
          geschrieben werden sollen. In der Langzeitsimulation werden die Werte automatisch durch '00' ersetzt.
          �nderungen in boden.f90 Z.1072ff., snow.f90 Z. 218 ff..
          Das �ffnen der *.lzs Dateien wurde �n der Kurzzeitsimulation hinzugef�gt (gebiet.f90 Z. 347ff.).
    \item Anfangswerte einlesen *.lzg\\
          Die Routine inp\_anf\_gerinne.f90 wurde so angepasst, dass das neue Format aus der *.lzg Datei gelesen werden
          kann. Um alte Modelle aufrecht zu erhalten, ist es weiterhin ohne weitere Angaben m�glich die Anfangswerte
          im alten Format einzulesen (Das Programm \glqq merkt \grqq selbst, um welches Format es sich handelt).
    \item Anfangswerte einlesen *.lzs\\
          Die Routine inp\_anf.f90 wurde so angepasst, dass das neue Format aus der *.lzs Datei gelesen werden
          kann. Um alte Modelle aufrecht zu erhalten, ist es weiterhin ohne weitere Angaben m�glich die Anfangswerte
          im alten Format einzulesen (Das Programm \glqq merkt\grqq  selbst, um welches Format es sich handelt).
\end{enumerate}

\textbf{Format�nderung- Darstellung Bodenfeuchte im GraphicTool (SK, 21.10.2004) (V2.0.2m2)}\\
Die Bodenfeuchte (Datei bof.dat, Kanalnummer nbof) konnte aufgrund der nicht Formatkonformen Ausgabe nicht mit dem
GraficTool angezeigt werden. Aus diesem Grund wurde das Ausgabeformat dem Blockformat angepasst. Es kann jedoch
trotzdem nur das erste Hydrotop angezeigt werden.(boden.f90 Z. 1367ff.)

\textbf{Faltung des Interflows mit ZFT anstelle des Parallelspeichers bei Berechnung mit Hydrotopen (SK, 21.10.2004) (V2.0.2m3)}\\
Der Interflow wird im Gebietsbearbeitungsfall ispk=7 (Hydrotope) mit der Zeitfl�chenfunktion (identisch zum
Oberfl�chenabflusses) gefaltet. Bei der Berechnung alter Modelle werden aus diesem Grund abweichungen in den
Ergebnissen vorhanden sein!(gebiet.f90 Z. 712ff.)

\textbf{Fehler in der Berechnung des CIN- und CEX-Wertes behoben (SK, EP, 21.10.2004) (V2.0.2m4)}\\
Der schon am 03.11.2003 festgestellte Fehler in der Berechnung des CIN und CEX-Wertes wurde jetzt endg�ltig behoben und
in die aktuelle Berechnungsroutine eingef�gt.\\
Zur Erinnerung des festgestellten Fehlers:\\
Beide Parameter wurden bislang in der Einheit [mm/dm/h] gefuehrt. In der Subroutine bodf\_n.f90 werden diese beiden
Parameter, die dort cin und cex hei�en in der Einheit [mm/h] erwartet. Der Fehler liegt darin, dass xbfm (max.
Bodenfeuchte) nur in der Einheit mm je dm Schichtdicke definiert ist. Diese Groesse muss daher noch mit der
tats�chlichen Schichtdicke multipliziert werden.

Da diese �nderung weitreichende Auswirkung auf das Berechnungsergebnis hat, gibt es Inkonsistenzen f�r alte geeichte
Modelle. Daher muss bei einer Berechnung alter Modelle an der alten, wenn auch fehlerhaften Definition von xcin und
xcex festgehalten werden. Bei alten Modellanwendungen also unbedingt an diese Umstellung denken!!! (inp\_bod\_b.f90 Z.
221ff.)

\textbf{Format der Ausgabendateien ge�ndert (JH, SK, 11.11.2004) (V2.0.2m5)}\\
F�r die Ausf�hrung mehrerer Berechnungen hintereinander wird die Bezeichnung der Ausgabedateien wieder mit dem
Anfangsdatum versehen. Die Endung .dat bleibt erhalten (z.B. 940313\_qgg.dat), die �nderungen von AG 18.08.02 werden
teilweise zur�ckgenommen. Au�erdem werden f�r einige Dateien die Kopfzeilen im Fall von synthetischen Niederschl�gen
angepasst. (kalypso-na.f90 Z. 790ff.)

\subsection{Version 2.0.2 zu Version 2.0.3}
\textbf{Anzahl der Zuflusszeitreihenwerte bei Kurzzeit auf 5000 erh�ht  (JH 22.05.2005) (V2.0.3m1)}\\
Liegt die Zuflusszeitreihe (Knoten zu Knoten Beziehung) im Grap-Format vor (Fall 5 in der
Netzdatei), so wird die Zeitreihe bei einer Kurzzeitsimulation mit der Subroutine
inp\_peg\_kz eingelesen. Hier wurde die m�gliche Anzahl an Datenpaaren auf 5000 erh�ht
(zuvor 1500), um den Simulationszeitraum bei Kurzzeitsimulation zu verl�ngern (bei 5min
sind ca. 17 Tage m�glich). (inp\_peg\_kz.f90 Z. 63)

\textbf{Bugfix: Inhalt Interzeptionsspeicher bei Hydrotopberechnung (JH, AvD 17.08.2005) (V2.0.3m2)}\\
Bei der Berechnung des Inhaltes des Interzeptionsspeichers wurde der Inhalt nicht richtig
initialisiert (es wurde immer der Wert des zuvor berechneten Hydrotopes als Anfangsinhalt
verwendet). In der Routine incept wurde die Initialisierung der Variablen bianf angepasst
und diese als Anfangswert verwendet. In der aufrufenden Routine boden wird die Variable
bianf als Liste in die Variable bi(nn) geschrieben, welche immer Hydrotopbezogen den
Inhalt des Interzeptionsspeicher vorh�lt. (incept.f90)

\textbf{Bugfix: ianz deklariert (JH, AvD 17.08.2005) (V2.0.3m3)}\\
In der Subroutine gebiet war die Variable ianz nicht deklariert. Diese wurde als Integer
deklariert. (gebiet.f90 Z. 134)

\subsection{Version 2.0.3 zu Version 2.0.4}
\textbf{Einf�hrung des Aufteilungsfaktors pro Abfluss  (JH 02.02.2006) (V2.0.4m1)}\\
Die Berechnung der Kalinin-Miljukov Parameter wird jetzt auf Basis der km-Dateien in
Kalypso Enterprise durchgef�hrt. In diesem Zuge der Umstellung wurde ein
abflussabh�ngiger Aufteilungsfaktor Fluss-Vorland eingef�gt. Au�erdem entf�llt die Angabe
des bordvollen Abflusses in der Kopfzeile jedes KM-Stranges in der Gerinnedatei, da jetzt
auch Str�nge berechnet werden k�nnen, welche den bordvollen Abfluss nie erreichen
(�nderung der Einleseroutine intrsp.f90). Die Berechnung der Abflussanteile anhand des
Aufteilungsfaktors wurde korregiert, da die alte Berechnung abh�ngig vom Bordvollen
Abfluss war und der Aufteilungsfaktor falsch verwendet wurde. Alte
Berechnung:$q_{Vorland}=(q-qfmax)\cdot(1-c)$. Neue Berechnung $q_{Vorland}=(q)\cdot(1-c)$
(gerinne.f90 Z.206ff.). Kommentar: Der Unterschied in der Berechnung ist beim Kollau
Modell verschwindend klein (jedoch sind hier die KM Parameter auch kalibriert und der
Bordvolle Abfluss meist nicht erreicht).

\subsection{Version 2.0.4 zu Version 2.0.5}
Die �nderungen betreffen die Einf�hrung von Mulden Rigolen (CK) und sind daher sehr
gro�r�umig angelegt.
\textbf{Format der Hydrotopdaten  (CK 19.04.2006) (V2.0.5m1)}\\
Die Hydrotopdaten werden jetzt formatfrei eingelesen und die nicht verwendeten Parameter
(x1, x2, x3, m\_hbod) wurden aus der Einleseroutine entfernt. Jetztiges Format:...
(inp\_hydro.f90)

\textbf{neue Module  (CK 19.04.2006) (V2.0.5m2)}\\
Es wurden die Module LOG\_FILE, NAConstants, ANF\_CONS und Physical\_CONS erstellt, um
hier die allgemeinen Parameter zu definieren. Das Modul LOG\_File dient zur Ausdabe des
Debug Log-Files. Dieses Log-File ist neu und wird nur im Debug ausgef�hrt (Parameter in
LOG\_File).

\textbf{Umstellung der Grundwasserberechnung (CK 19.04.2006) (V2.0.5m3)}\\
Die grundwasserberechnung erfolgt jetzt in einer eigenen Routine  auf Basis der
Teilgebiete (der Teil wurde vollst�ndig aus der Routine boden herausgel�st). Die neue
Routine hat den namen gwsp (welches nichts mit der vorhandenen Routine zu tun hat - d.h.
die alte gwsp Routine wurde gel�scht, da diese nicht mehr verwendet wurde).

\textbf{Ispk=7 (CK 19.04.2006) (V2.0.5m4)}\\
Im Code wird nurnoch der Gebietsbearbeitungfall ispk=7 unterst�tzt (Berechnung auf Basis
von hydrotopen).

\textbf{Berechnung der Evaporation in neuer Methode (CK 19.04.2006) (V2.0.5m5)}\\
Die Evaporation wird in der neuen Methode evapot durchgef�hrt.

\textbf{Mulden Rigolen (CK 19.04.2006) (V2.0.5m6)}\\
Die Berechnung der Mulden Rigolen erfolgt in der Routine qmr, welche neu hinzugef�gt
wurde und �hnlich zur bodenroutine funktioniert. Zum Einlesen der Mulden Rigolen Angaben
in der Datei *.mr ist die neue Routine inp\_mrs erstellt. Die Ausgabe der Ergebnisse
(Muldenrigolenabfluss) erfolgt in der Datei ...qmr.dat. Hierzu wurde die Ausgabedatei in
der Konfiguration hinter der Evapotranspiration eingef�gt. Die Mulden Rigolen Berechnung
funktioniert, wurde jedoch noch nicht ausreichend getestet.

\subsection{Version 2.0.5 zu Version 2.0.6}
Die �nderungen betreffen die Einf�hrung von Mulden Rigolen (CK) und sind daher sehr
gro�r�umig angelegt.
\textbf{Einf�hrung des Loggers  (JH) (V2.0.6m1)}\\
Zur besseren fehlerausgabe f�r das User Interface wurde ein Interface implementiert,
welches die Fehlermeldungen als XML Datei ablegt. Die �nderungen beziehen sich somit auf
alle im Code vorhandenen Fehlermeldungen (die alten write statements wurden nicht
entfernt). (logger.f90)

\textbf{�nderung der Kalinin Miljukov Parameter  (JH 19.01.2007) (V2.0.6m1)}\\
Es ist nicht mehr erforderlich ,dass der 3te Abfluss der bordvolle Abfluss ist. Die
Abgrenzung Fluss- Vorland erfolgt anhand des Aufteilungsfaktors alpha. (gerinne.f90)


\subsection{Version 2.0.6 zu Version 2.0.7}
Die �nderungen betreffen die Einf�hrung von van genuchten Parametern und Makroporen (KV)
und sind daher sehr gro�r�umig angelegt.

\subsection{Version 2.0.7 zu Version 2.0.8}
Die �nderungen betreffen haupts�chlich die Umstellung auf den g95 Compiler, sowie
allgemeines Aufr�umen des Codes. Um den Code �bersichtlicher zu gestalten wurden viel
nicht verwendete Coderudimente entfernt. Dies beinhaltet, dass lediglich noch die von dem
User Interface verwendeten Formate unterst�tzt werden!
\textbf{Grap Format eingeschr�nkt  (JN 14.06.2007) (V2.0.7m1)}\\
Das grap Format f�r die Kurzzeitniederschl�ge wird lediglich in der von der GUI
verwendeten Form ber�cksichtigt. Die Variablen ianz, fl, dtc, inum werden
im Format nicht ben�tigt und daher auch nicht mehr gelesen.
\begin{verbatim}
Kommentartext
       198001010000   0  0.      0.
grap
28.03.1997 07:09:00 0
28.03.1997 07:10:00 0.1
28.03.1997 07:29:00 0
28.03.1997 07:30:00 0
\end{verbatim}
(inp\_nied\_kz\_dat.f90)

\subsection{Version 2.0.8 zu Version 2.1.0}
Es handelt sich um die Zusammenf�hrung der Version 2.0.8 mit den Programmierungen
von Christoph K�pferle zur Berechnung der Mulden-Rigolen. Bei Vorliegen einer *.mr Datei Werden
Mulden-Rigolen berechnet. Die Bodendatei ben�tigt zwei parameter vmacro und r, die f�r die van Genuchten Berechnung 
von Karolina gebraucht werden. 


\begin{verbatim}
Kommentartext
Kommentartext: TG-Nummer
Kommentartext: Fl�che[m^2]  Nutzung  Bodenprofil  max.Perk[mm/d] Zufluss-GW[%]   
Kommentartext: TG-Nummer	Anzahl Elemente
Kommentartext: Hydrotop-ID d-Dr�nrohr[mm] kf-Dr�nrohr[mm/d] Gef�lle-Dr�nrohr
[prommille] Rahigkeit-Dr�n [mm] Muldenbreite[m]	entw�. Knoten
4500
580. MRS_N mrs 2.8E-8 1.0
200. 4270. 0.003 2. 1.8 0
\end{verbatim}

\begin{verbatim}
Typ       Tiefe[dm] vmacro r
mrs          4
mulde   3.0 0.0 0.00 0.00
filter  3.0 0.0 0.00 0.00
mSfsld2 7.0 0.0 0.00 0.00
mSfsld2 1.0 0.0 0.00 0.00
mS-Sl2_3     2
Hnz4SV3  8.0 0.0 0.00 0.00
mS-Sl22  10.0 0.0 0.00 0.00
\end{verbatim}

\newpage
\section{ToDo-Liste}
\begin{itemize}
    \item Umstellung auf g95 Compiler
    \item arrays dynamisieren (allocatable arrays nutzen?!?)
    \item Korrekturfaktoren/ Gleichungen pr�fen
    \item 0 Uhr / 24 Uhr Problem bei Ausgabe
    \item Fortsetzung des Modelles (f�r Automatische Kalibrierung)
\end{itemize}




%
%********** A N H A N G ********************************************************************************************************************************************
%





%\input{literaturaufruf}
\bibliographystyle{gerplain}
\bibliography{literatur}


\begin{appendix}
\newpage
%\input{}
\end{appendix}


\end{document}
